\newpage 
\chapter{The Standard Model of Particle Physics}
%\section{The Standard Model of Particle Physics}

%To pave the way for our future studies we present the SM. Complete with a overview of it's mechanisms and a brief historical introduction.
%

\section{Motivation}

Has stated, it is hard to question the validity of the SM as a successful, at least approximate, framework with whom to describe the phenomenology of Particle Physics up to the largest energy scales probed by collider measurements so far although some inconsistencies must be addressed.  
%
The SM was proposed in the nineteen sixties by Glashow, Salam and Weinberg and since it has been extensively tested. Both in contemporary direct searches for new physics and indirect probes via e.g. flavour anomalies and precise electroweak parameter measurements in proton-electron collisions, { \color{gray} and as said, it's been consistent with most to date.} 

The path to the formulation of the SM came from previous principles relating to symmetries in nature, specificity symmetry in physical laws. In fact much in modern physics can be attributed to Emmy Noether. Who deduced trough her first theorem that if the action in a system is invariant under some group of transformations (symmetry), then there exist one or more conserved quantities (constants of motion) which are associated to these transformations. 

This led to the fundamental question behind the SM. Is it possible that upon imposing to a given Lagrangian the invariance under a certain group of symmetries to reach a given form of the dynamics. These dynamics would be particle interactions and this train of thought led to Quantum Electrodynamics (QED).

We can quote Salam and Ward: % A. Salam and J. C. Ward, Nuovo Cim.19, 165 (1961). 

\textit{“Our basic postulate is that it should be possible to generate strong,  weak and electromagnetic  interaction terms (with all their correct symmetry properties and also with clues regarding their relative strengths) by making local gauge transformations on the kinetic energy terms in the free Lagrangian for all particles.”}

We are glossing over a lot of complexity here, and for the SM to be properly formulated other concepts would required. In the case of weak interactions, the presence of very heavy weak gauge bosons require the new concept of spontaneous breakdown of the gauge symmetry and the Higgs mechanism. While 
% [63,  64,  65].  
% [63]  P. W. Higgs, Phys. Lett.12, 132 (1964).
% [64]  F. Englert and R. Brout, Phys. Rev. Lett.13, 321 (1964).
% [65] G. S. Guralnik, C. R. Hagen, and T. W. B. Kibble, Phys. Rev. Lett.13, 585 (1964).
%
the concept of asymptotic freedom played a crucial role to describe perturbatively the strong interaction at short distances.  
% 
% [89, 90]
% [89]  D. J. Gross and F. Wilczek, Phys. Rev. Lett.30, 1343 (1973).
% [90]  H. D. Politzer, Phys. Rev. Lett.30, 1346 (1973). 

\section{Internal symmetry of the Standard Model}

The SM is "standard" QFT gauge theory, that is to say, that it is manifestly invariant under a set of field transformations. The group on which it's based on is widely known and seen in, 

\begin{equation}
\mathrm{SU}(3)_{\mathrm{c}} \times \SU{L} \times \U{Y} \quad  .
\label{eq:SM Group}
\end{equation} 

First, the $\mathrm{SU}(3)_{\mathrm{c}}$ group corresponding to quantum chromodynamics (QCD) responsible for the strong force,  we'll see this group remains unbroken, while after we have the $\SU{L} \times \U{Y}$ portion that will be broken by the Higgs mechanism into $\U{Q}$ the electromagnetic gauge symmetry. Each particle stems from a field that is charged in a particular manner on each of these groups, making the charge triplets we will come to later define. Given the invariance under this group we'll see that we cannot have any field that is charged have a explicit mass term. { \color{gray} All masses for the fermions and leptons are generated trough their interactions with the Higgs Boson. This mass generation as the Higgs Boson settles into it's VEV is called the Higgs Mechanism. } 

\subsection{Fields and Lagrangian}

The Standard Model spectra after the process SBB is composed by, first, the weak force carriers, gauge bosons $W^\pm$ and $Z$, and the photon $\gamma$, the electromagnetic interaction messenger and finally the strong force mediators, the gluons, $g$, as well, of course, by the matter particles, the quarks and leptons. 

Fermions and quarks are organized in three generations each, with 2 pairs by each generation leading to 6 different particles for each family. For quarks we have the up and down for the first generation, charm and strange for the second as well as top and bottom for the third one. Similarly, there are 6 types of leptons, the charged ones, electron, muon and tau, and the associated neutrinos. These are represented in different manners, being that the quarks are represented by the letters $(u,d,c,s,t,b)$ while leptons as $(e,\nu_{e},\mu,\nu_{\mu},\tau,\nu_{\tau})$. 

Fermions are half integer spin particles most of which have electrical charge (except the neutrinos).  While quarks interact via the weak, electromagnetic and strong forces, the charged leptons only feel the electromagnetic and weak forces and the neutrinos are weakly interacting.  

A physical fermion is composed of a left-handed and a right-handed field. While the left transform as $SU(2)_L$ doublets and can be written as,

\begin{equation}
L^i= \begin{pmatrix}
\nu_{e_L} \\ e_L 
\end{pmatrix},
\begin{pmatrix}
\nu_{\mu_L} \\ \mu_L 
\end{pmatrix},
\begin{pmatrix}
\nu_{\tau_L} \\ \tau_L 
\end{pmatrix} 
\quad 
\text{and} \quad Q^i= \begin{pmatrix}
u_{L} \\
d_L 
\end{pmatrix},\begin{pmatrix}
c_{L} \\
s_L 
\end{pmatrix}
,\begin{pmatrix}
t_{L} \\
b_L 
\end{pmatrix} \quad ,
\end{equation}
where the $i$ index stands for generation, often designed as the flavour index, the latter are $\SU{L}$ singlets and can be simply represented as
%
 \begin{equation}
e^i_R=\{e_R,\mu_R,\tau_R\}, \quad  u^i_R=\{u_R,c_R,t_R\}, \quad d^i_R=\{d_{e_R},s_{e_R},b_{e_R}\} \quad , 
\end{equation}
%
note also that the quarks form triplets of $\mathrm{SU(3)_C}$ whereas leptons are colour singlets. The Higgs boson also emerges from an $\mathrm{SU(2)_L}$ doublet with the form,
%
\begin{equation}
H=\begin{pmatrix}
\phi^1 + \; i \; \phi^2 \\
\phi^3 + \; i \; \phi^4  
\end{pmatrix} \quad , 
\end{equation}
%
%
%The Lagrangian that describes all vector particles and gauge fields in the SM can be writen as
%
Here we see the four components that correspond to the respective degrees of freedom of the Higgs Field. 

The full set of quantum numbers in all the SMs fields are described in the tables \ref{table1} and \ref{table2}, with the color charge, weak isospin number and the hypercharge are given by their ordered entries in each triplet.
%
\begin{table}[H]
\centering
\caption{Gauge bosons and Scalar fields in the SM}
\label{table1}
\begin{tabular}{@{}cccccc@{}}
  \hline	
 Fields & Spin 0 field & Spin 1 Field & $SU(3)_C \times SU(2)_L \times U(1)_Y$  \\
  \hline	
 Gluons  & $\times$  & $g$ & (8,1,0) \\	
A bosons & $\times$  & $A^i$ & (1,3,0)   \\
B bosons & $\times$  & $B$ & (1,1,0)   \\
Higgs field & ($\phi^\pm, \phi^0 )$  & $\times$ & (1,2,1) \\ \hline
\end{tabular}
\end{table}
%~
\begin{table}[H]
\centering
\caption{Fermion field dimensions in the SM}
\label{table2}
\begin{tabular}{@{}cccccc@{}}
  \hline	
 Fields & Spin $\frac{1}{2}$ Field & $SU(3)_C \times SU(2)_L \times U(1)_Y$  \\
  \hline	
Quarks (3 gen.) & $Q=(u_L,d_L)$ & $(3,2,\frac{1}{3})$ \\	
$\quad$        & $u_R$ & $(3,1,\frac{4}{3})$   \\
$\quad$   & $d_R$ & $(3,1, -\frac{2}{3})$   \\
Leptons (3 gen.) & $(L=(\nu_{e_L}, e_L )$ & $(1,2,-1)$  \\
$\quad$   & $e_R$ & $(1,1,-2)   $ \\ \hline

\end{tabular}
\end{table}
%
From here, given the gauge group in, eq.\,\ref{eq:SM Group} and accounting for the charges and fields, we can derive the form of the SM's Lagrangian. We thus begin with the discussion of the 12 generators of our SM gauge group, we write their algebra as, 
% 
\begin{equation}
\left[ L_a , L_b \right] = i f_{abc} L_c \quad \left[ T_a , T_b \right] = 1 \epsilon_{abc} T_c \quad \left[ L_a , T_b \right] = \left[ L_a , Y \right] = \left[ T_b,Y \right] = 0 
\end{equation}
%
where for the $\mathrm{SU(3)_c}$ triplets, $L_a= \frac{\lambda_a}{2}$ , $(a = 1, . . . , 8)$ {\color{gray} contrary to $\mathrm{SU(3)_c}$ singlets where, $L_a = 0$}.  As for the $\mathrm{SU(2)_L}$, we have $T_i= \frac{\sigma_i}{2} $, $(i = 1, 2, 3)$, {\color{gray} being that again for singlets $T_b=0$}. $Y$ is the generator of $U(1)_Y$. The symbols $\lambda_a$ and $\sigma_i$ represent the Gell-Mann and Pauli matrices respectively. Given the SM gauge groups the covariant derivative, $D_\mu$, will read as, 
%
\begin{equation}
\label{eq:PartialDefSM}
D_\mu = \partial_\mu - i g_S \tau^a G^a_\mu - i g T^i A^i_\mu - i g' Y B_\mu \quad ,  
\end{equation}  
%
We can expect 3 different type of couplings, $g_s$ related to the $\mathrm{SU(3)_C}$ subgroup, $g$ to the $\mathrm{SU(2)_L}$ and $g^\prime$ to $\mathrm{U(3)_Y}$. The associated canonical field strength tensors would be,
\begin{align}
G_a^{\mu \nu} & = \partial^\mu G^\nu_a - \partial G^\mu_a - g_s f_{abc} G_a^\mu G_b^\nu  \\ 
W_a^{\mu \nu} & = \partial^\mu W^\nu_a - \partial^\nu W^\mu_a  - g \epsilon_{abc} W^\mu_b W^\nu_c \\
B^{\mu \nu}   & = \partial^\mu B^\nu - \partial^\nu B^\mu 
\end{align}
It is often convenient to present the SMs Lagrangian in portions, usually divided in three sections,
\begin{equation}
\mathcal{L}_{SM} = \mathcal{L}_{kin}  +  \mathcal{L}_{Yuk} +  \mathcal{L}_{\phi} 	
\end{equation}
Where we have the kinetic portion of the SM terms,$\mathcal{L}_{kin}$, responsible for  free propagation of particles, the Yukawa portion, $\mathcal{L}_{Yuk}$  corresponding to interactions of particles with the Higgs Boson, and finally the $\mathcal{L}_{\phi}$ scalar potential. The full kinetic portion of the SM read, 
%
\begin{align}
\label{eq:KinSM}
\mathcal{L}_{kin} = & - \frac{1}{4} G^{\nu \mu}_a G_{a \,\nu \mu}  - \frac{1}{4}  W^{\nu \mu}_a W_{a \,\nu \mu}  
- \frac{1}{4}  B^{\nu \mu} B_{\nu \mu} \nonumber \\ 
 & -i \overline{Q_{L_i}} \slashed{D} Q_{L_i} 
   -i \overline{u_{R_i}} \slashed{D} u_{R_i}  
   -i \overline{d_{R_i}} \slashed{D} d_{R_i}  
   -i \overline{L_{L_i}} \slashed{D} L_{L_i}    
   -i \overline{e_{R_i}} \slashed{D} e_{R_i}   \\
 & - (D_\mu H)^\dagger ( D^\mu H )   \nonumber 
\end{align}
Where $\slashed{D}$ is the Dirac covariant derivative, $\gamma^\mu D_\mu$. { \color{gray} From the last line Eq.\,\ref{eq:KinSM} and with Eq.\,\ref{eq:PartialDefSM} we will present how the Generators $A^i_\mu$ and $B_\mu$ give rise to the weakly interacting vector bosons $W^\pm$ and $Z^0$ and the electromagnetic vector boson $\gamma$. Contrary to the color sector, where the eight generators $G^a_\mu$ simply correspond to eight gluons $g$ a mediating strong interactions. } { \color{blue} Maybe move this to where it actually is written } 
%
While the scalar potential part 
%
\begin{equation}
\mathcal{L}_{\phi} = -\mu^2 H H^\dagger - \lambda (H H^\dagger)^2
\end{equation}
Finally the Yukawa portion of the Lagrangian would be written as, 
\begin{equation}
\label{eq:YukawaSM}
\mathcal{L}_{Yuk} = Y^u_{ij} \overline{Q_{L_i}} u_{R_j}  \tilde{H} + Y^d_{ij} \overline{Q_{L_i}}  d_{R_j} H  + Y^e_ij \overline{L_{L_i}}  e_{R_i} H + h.c. 
\end{equation}
%
Here we have, $\tilde{H}=i\sigma_2 H$.
%
{ \color{gray} It is due to the Yukawa interactions between the Higgs and the fermions and leptons that these acquire their masses once the Higgs settles into his VEV. } { \color{blue} same as before }

\section{The Higgs mechanism and the mass generation of the Gauge bosons}

From what was defined above, we can now study the process SSB by which, 
\begin{equation}
\SU{L}\times\U{Y} \rightarrow \U{Q}
\end{equation} and carry trough to the Higgs Mechanism. Enabling us to find the real physical states of the gauge bosons and the origin of their mass. Let us then consider the part of the Lagrangian containing the scalar covariant derivatives, the scalar potential and the gauge-kinetic terms:
%
\begin{equation}
\mathcal{L}_{Gauge} \supset (D_\mu H)(D^\mu H)^\dagger - \mu^2 H^\dagger H - \lambda (H^\dagger H)^2 - \frac{1}{4}  W^{\nu \mu}_a W_{a \,\nu \mu}  
- \frac{1}{4}  B^{\nu \mu} B_{\nu \mu}
\label{eq:GaugeSM}
\end{equation} 
% 
We expect a phase shift to occur, namely one that ensures $\mu^2 < 0$ while at the same ensuring that the field now explicitly breaks the $\mathrm{SU(2)_L \times U(1)_Y}$. For this to happen we expect the shifted squared value of the Higgs field to be,
%
\begin{equation}
(H^\dagger H) = \frac{-\mu^2}{2\lambda} = \frac{1}{2} v  \quad , 
\end{equation} 
This VEV, called the electroweak VEV, is experimentally measured to be $v \approx 246$ GeV. 
%
The choice of vacuum can be aligned in such a way that we have,
\begin{equation}
H_{min} = \frac{1}{\sqrt{2}} \begin{pmatrix} 0 \\
v 
\end{pmatrix} \quad .
\end{equation}

Given that now the $SU(2)_L \times U(1)_Y$ symmetry is broken down to $U(1)_Q$ we jump from a scenario where there were four generators, which are $T^{1,2,3}$ and $Y$, to, after the breaking, having solely one unbroken combination that is $Q =  (T^3 + 1/2)$ associated to the electric charge. This means that in total we will have three broken generators, thus, from Goldstone Theorem, there would have to be created three massless particles. 

These Goldstones modes however can then be parametrized as phases in the field space and then can be "rotated away" in the physical basis, leaving us with a single physical massive scalar, the Higgs boson. Note that, with this transformation we are removing three scalar degrees of freedom.  However, they cannot just disappear from the theory and will be absorbed by the massive gauge bosons.
%
In fact, a massless gauge boson contains only two scalar degrees of freedom (transverse and polarization). Meanwhile, a massive vector boson has two transverse and a longitudinal polarization, i.e., three scalar degrees of freedom. So, as we discussed above, while before the breaking of the EW symmetry we have four massless gauge bosons, after the breaking we are left with three massive ones. This means that there are three extra scalar degrees of freedom showing up in the gauge sector. It is then commonly said that the goldstone bosons are ``eaten" by the massive gauge bosons and the total number of scalar degrees of freedom in the theory is preserved. Therefore, without loss of generality, we can rewrite the Higgs doublet as
%
\begin{equation}
 \begin{pmatrix}
G_1 + i G_2 \\ 
v + h(x) + i G_3 
\end{pmatrix} = H (x) \rightarrow H (x) =  \frac{1}{\sqrt{2}} \begin{pmatrix}
0 \\ 
v + h(x) 
\end{pmatrix} \quad .
\label{shame}
\end{equation}
Once the Higgs doublet acquires a VEV, the Lagrangian (\ref{eq:GaugeSM}}) can be recast as:
\begin{align}
\mathcal{L}^\prime = & \frac{1}{2} \partial_\mu h \partial^\mu h - \frac{1}{2} (2v^2 \lambda) h^2
 - \frac{1}{4}  W^{\nu \mu}_a W_{a \,\nu \mu}  
- \frac{1}{4}  B^{\nu \mu} B_{\nu \mu}  \nonumber \\
& + \frac{1}{8} v^2 g^2 (A^1_\mu A^{1,\mu}+ A^2_\mu A^{2,\mu}) +  \frac{1}{8} v^2  (g^2  A^3_\mu A^{3,\mu} + g^{\prime 2} B_\mu B^\mu - 2 g^2 g^{\prime 2} A^3_\mu B^\mu ) \quad , 
\label{complicatedpart}
\end{align}
A few things become obvious first, we have a lot of mass terms most stemming from the squared gauge fields and a lonesome squared mass term belonging to the real scalar field we know to be the Higgs field. This makes the Higgs boson mass in the SM to be given by,
%
\begin{equation}
M_h= (2v^2 \lambda) \quad .  
\end{equation}
%
To obtain masses for the gauge bosons we need to rotate the gauge fields to a basis where the mass terms are diagonal. First, it is straightforward to see that the electrically charged eigenstates are given by %\ref{gagestate}
%First the fields that carry defined charge that can be easily shown to be 
\begin{equation}
W^\pm_\mu = \frac{1}{\sqrt{2}} (A^{(1)}_\mu \pm i A^{(2)}_\mu) \quad , 
\label{gagestate}
\end{equation}
meaning that the mass of the W bosons is, 
\begin{equation}
M_{W^\pm}= \frac{1}{2} v g \quad .
\end{equation}
The situation becomes a bit more complicated for the second term in (\ref{complicatedpart}) due to a mixing between $A_\mu^3$ and $B_\mu$. In the gauge eigenbasis the mass terms read
\begin{equation}
\begin{pmatrix}
A_\mu^3 && B_\mu
\end{pmatrix} \cdot  \frac{1}{4} \nu ^2 \begin{pmatrix}
g^2  & -g g^\prime \\
-g g^\prime & g^{\prime 2} 
\end{pmatrix} \cdot \begin{pmatrix}
A_\mu^3 \\  B_\mu
\end{pmatrix}  \quad , 
\end{equation} 
which can be diagonalized to obtain
\begin{equation}
\begin{pmatrix}
A_\mu && Z_\mu 
\end{pmatrix} \begin{pmatrix}
0  & 0 \\
0  & \frac{1}{2} v \sqrt{g^2 + g^{\prime 2}} 
\end{pmatrix}  \begin{pmatrix}
A^\mu \\ Z^\mu
\end{pmatrix}  \quad , 
\end{equation}
%Where the new eigenvectors that represent the $Z$ boson and the photon, $A^\mu$ in terms of the former base are written as,
%
we identify the eigenvector associated to the eigenvalue 0 to the photon and the massive one, $ M_Z =  \frac{1}{2} v \sqrt{g^2 + g^{\prime 2}} $, to the Z boson. Such eigenvectors can be written as
%
\begin{align}
A_\mu &=\cos(\theta_\omega) B_\mu + \sin(\theta_\omega) A_\mu^3 \quad ,  \\  
Z_\mu & =- \sin(\theta_\omega) B_\mu + \cos(\theta_\omega) A_\mu^3 \quad , 
\end{align}
%
where $\theta_w$ is the so called Weinberg mixing angle and is defined as, 
\begin{equation}
\cos(\theta_\omega)=\frac{g}{ \sqrt{g^2 + g^{\prime 2}}} \quad , 
\end{equation}
thus clearly showing the massless photon along with a massive Z boson with mass $M_Z= \frac{1}{2} \nu \sqrt{g^2 + g^{\prime 2}} $. 
%
So we conclude our exploration of the electroweak sector with all the correct massive spectrum observed and its origin discussed.


\section{Fermion Masses, Yukawa Couplings and the CKM Matrix}

Given the charges of the fermion and lepton fields we cannot construct a gauge invariant theory with explicit mass terms for these fields. As mentioned, the mass of these particles will have to be generated by the Higgs Mechanism. This is easy to demonstrate as by returning to eq.\,\ref{eq:YukawaSM}, which is rewritten, 
\begin{equation}
\label{eq:YukawaSM2}
\mathcal{L}_{Yuk} = Y^u_{ij} \overline{Q_{L_i}} u_{R_j}  \tilde{H} + Y^d_{ij} \overline{Q_{L_i}}  d_{R_j} H  + Y^e_ij \overline{L_{L_i}}  e_{R_i} H + h.c. 
\end{equation}
%
We define the yukawa matrices, $Y^{e,u,d}$ as generic $3\times3$ complex and hermitian coupling matrices without any dimensions and keep $H$ as previously defined. Here also as alluded to previously, we have the flavour indices, $i,j=\{ 1,2,3 \}$. By shifting the Higgs field $H$ as in Eq. (\ref{shame}), { \color{gray} expansion of all  terms we obtain, 
%
\begin{align}
\mathcal{L}_y = \, \, & y^d 
\begin{pmatrix}
\overline{u_L} & \overline{d_L} 
\end{pmatrix} d_R 
\begin{pmatrix}
0\\ v + h(x)
\end{pmatrix}  + 
y^s 
\begin{pmatrix}
\overline{c_L} & \overline{s_L} 
\end{pmatrix} s_R 
\begin{pmatrix}
0 \\ v + h(x)
\end{pmatrix} \nonumber  \\ & +
 y^b 
\begin{pmatrix}
\overline{t_L} & \overline{b_L} 
\end{pmatrix} b_R 
\begin{pmatrix}
0 \\ v + h(x)
\end{pmatrix}  +
 y^u 
\begin{pmatrix}
\overline{u_L} & \overline{d_L} 
\end{pmatrix} d_R 
\begin{pmatrix}
v + h(x) \\ 0
\end{pmatrix} \nonumber \\ + &   y^c 
\begin{pmatrix}
\overline{c_L} & \overline{s_L} 
\end{pmatrix} d_R 
\begin{pmatrix}
v + h(x) \\ 0 
\end{pmatrix} +
y^t
\begin{pmatrix}
\overline{t_L} & \overline{b_L} 
\end{pmatrix} t_R \begin{pmatrix}
 v + h(x) \\ 0
\end{pmatrix}   \\ + &  y^e 
\begin{pmatrix}
\overline{\nu_{e_L}} & \overline{e_L} 
\end{pmatrix} e_R 
\begin{pmatrix}
0 \\ v + h(x)
\end{pmatrix} +
 y^\mu 
\begin{pmatrix}
\overline{\nu_{\mu_L}} & \overline{\mu_L} 
\end{pmatrix} \mu_R
\begin{pmatrix}
0 \\ v + h(x)
\end{pmatrix} \nonumber  \\  + & 
 y^\tau \
\begin{pmatrix}
\overline{\nu_{\tau_L}} & \overline{\tau_L} 
\end{pmatrix} \tau_R 
\begin{pmatrix}
0 \\ v + h(x)
\end{pmatrix} +  h.c  \nonumber \quad , 
\end{align}
}
%
Further expansion of these terms would result equations like, \textit{e.g.} in the electron's case,  
%
\begin{equation}
\mathcal{L}_{y_e} = \frac{1}{\sqrt{2}} y^e \; v \left( \overline{e_L} e_R + \overline{e_R} e_L \right) + \frac{1}{\sqrt{2}} y^e h(x) \left( \overline{e_L} e_R + \overline{e_R} e_L \right) \quad , 
\label{electron mass}
\end{equation}
%
This clearly translates to the first terms in (\ref{electron mass}) equating to a mass term, $m_e = y^e v $. While the second terms represent interaction coupling between the electron and the Higgs boson.  
%
This is how the Higgs mechanism generates the mass to all of the fermionic and leptonic sector except neutrinos due to the SM not containing right handed neutrinos. Now it is beneficial to write the mass terms as matricial form as,

The lack of a right handed neutrinos is very commonly addressed in most BSM scenarios and in fact we'll see that their existence is account for in the models proposed further ahead in this work. 

Generically we can write these terms as,
\begin{equation}
M^u_{ij} = Y^u_{ij} \frac{v}{\sqrt{2}} \quad M^d_{ij} = Y^d_{ij} \frac{v}{\sqrt{2}} \quad M^e_{ij} = Y^e_{ij} \frac{v}{\sqrt{2}}
\end{equation}
The diagonalization of the mass matrices is trivial and expected, however note the Yukawa matrices remain arbitrary to some extent. The Yukawa matrices contain some parameters that are not observable since they can be absorbed by redefinitions of the real fermion fields i.e. there are several combinations of $Y^{u,d}$ that return the same mass spectrum. 

Note that the increasing masses seen in each generation depend directly on the term hierarchy of the Yukawa terms. This means that the mass of all particles directly relate to how strongly they each interacts with the Higgs boson. If you then take into account the real masses i.e. for the leptons and you'll deduce that given that the tau mass being in the GeV range while the electron in the 0.1 MeV range. These represent a very different coupling for each generation to the Higgs.  

\subsection{Cabibbo–Kobayashi–Maskawa matrix}

{ \color{red} I need to do more research on this chapter, I need to better explain why we don't have FCNCs at tree level. }  

The interactions of quarks are defined from experimental results, so it is required that the SM can both explain the mass spectrum of quarks and the measured Cabibbo Kobayashi Maskawa (CKM) matrix. The CKM matrix originates from messurements of charged flavour changing currents (FCNCs).

From the lack of FCNCs observed (at non-perturbative orders) the diagonalization of both $M^{u,d}$ and $Y^{u,d}$ simultaneously should be possible. {{ \color{blue} The physical interpretation of this would naturally be that the gauge and mass basis are equal } and no off-diagonal interactions exist between generations of fermions. This diagonalization would generically be achieved through a bi-unitarity transformation, 

\begin{align}
\begin{split}
m^u_{diag.}  =  & \frac{v}{\sqrt{2}} \mathrm{U}^u_L \mathrm{Y} {\mathrm{U}^u_R}^\dagger \nonumber \\ 
m^d_{diag.}  = & \frac{v}{\sqrt{2}}  \mathrm{U}^d_L \mathrm{Y} {\mathrm{U}^d_R}^\dagger 
\end{split}
\end{align}

Naturally we can invert these equations such that, 
\begin{align}
\label{eq:YukawaBiUni}
\begin{split}
Y^u_{ij} = & \frac{\sqrt{2}}{v} (U_L^u m^u_{diag.} U_R^u)_{ij} \\
Y^d_{ij} = & \frac{\sqrt{2}}{v} (U_L^d m^d_{diag.} U_R^d)_{ij}
\end{split}
\end{align}
If we examine the effect this change had on the mass terms of quark fields, by replacing the result of eq.\,\ref{eq:YukawaBiUni} in eq\, \ref{eq:YukawaSM}.

\begin{align}
\begin{split}
\mathcal{L}_{Yuk} \supset -\frac{\sqrt{2}}{v} d_{L\,i} \, d_{R\,j} & - \frac{\sqrt{2}}{v} u_{L\,i} \, u_{R\,j} + h.c. \\ 
 & \Downarrow  \\
-(U_L^d m^d_{diag.} U_R^d)_{ij} d_{L\,i} \, d_{R\,j}  &- (U_L^u m^u_{diag.} U_R^u)_{ij} u_{L\,i} \, u_{R\,j} \\ 
& \Downarrow \\ 
-m^d_{diag.} d_{L\,i}^\prime \, d_{R\,j}^\prime & - m^u_{diag.} u_{L\,i}^\prime \, u_{R\,j}^\prime
\end{split}
\end{align}

where the primed fields are the quark fields in the mass basis, defined by our transformations as, 
\begin{equation}
\begin{split}
d^\prime_{L,R} = U^d_{L,R} d_{L,R} \\
u^\prime_{L,R} = U^u_{L,R} u_{L,R} 
\end{split}  
\end{equation}
%
However this definition carries with it a problem, remember that while the left handed quark fields are $SU(2)_L$ doublets their right handed counter parts are not. So the coupling of the of fermions to their respective gauge fields changes, if we expand the up and down quark fields on the section of the Lagrangian,
\begin{align}
\begin{split}
\mathcal{L}_{ferm} \supset & 
\frac{1}{2} \overline{u^\prime_L} \gamma^\mu \left( g^\prime Y_L B_\mu + g W^0_\mu  \right) \left(U^u_L U^{u \dagger}_L \right) u^\prime_L - \frac{1}{\sqrt{2}} g \overline{u^\prime_L} \gamma^\mu \left( U^u_L U^{d \dagger}_L \right) d^\prime_L W^+_\mu \\ \nonumber   
- 
& \frac{1}{\sqrt{2}} g d^\prime_L \gamma^\mu \left( U^u_L U^{d \dagger}_L \right) u^\prime_L W^-_\mu 
+ 
\frac{1}{2} \overline{d^\prime_L} \gamma^\mu \left( g^\prime Y_L B_\mu - g W^0_\mu \right) \left( U^d_L U^{d \dagger}_L \right) d^\prime_L  
\end{split}
\end{align}
Through the use of the unitary property of these transform operators, $ \mathrm{U}^{u,d}_{L,R} \mathrm{U}^{u,d \dagger}_{L,R} = 1$, we can see that the interactions with the neutral bosons remain the same in the mass basis, however we can see that the charged currents are affected by this change. Therefor we define the the CKM matrix, as $V_{CKM} = U^u_L U^{u ^\dagger }_R $ and write the sensitive terms,
\begin{equation}
\mathcal{L}_{FCNCs} \supset \frac{1}{\sqrt{2}} g \overline{u}^\prime_L \gamma^\mu V_{CKM} d_L^\prime W^+_\mu + h.c. 
\end{equation}
This is in fact a very interesting feature of the Standard Model, by consequence of the $\mathrm{SU(2)_L \times U(1)_Y }$ symmetry there are no interactions of the right handed unitary matrices and there for no mixing, coupling, or charged currents of right handed quarks, making them theoretically invisible to measurements.  

% PERGUNTAR Note also that the mass term for every lepton depends directly on each of the respective and different Yukawa term and that the SM is a theory where the Yukawa matrices are diagonal by nature not having inter generational terms. {\color{blue} Is it doe? what about FCNCs?}
%

{\color{red} I am having a real hard time explaining the CKM part only affecting the left handed and doing the entire demonstrations I need help}

{\color{green} 
\subsection{Quantum chromodynamics}
I should meantion the QCD part of the Lagrangian 

I should explain what is asymptoic freedom and meantion that the strong force fades fast. Free quarks at high energies}

 
