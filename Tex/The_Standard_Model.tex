\newpage 

\section{The Standard Model of Particle Physics}

To pave the way for our future studies we present the SM. Complete with a brief overview of it's mechanisms and a historical introduction.
%
Has stated, it is hard to question the validity of the SM as a successful approximate framework with whom to describe the phenomenology of Particle Physics up to the largest energy scales probed by collider measurements so far. 
%
The SM was proposed in the sixties by Glashow, Salam and Weinberg and since it has been extensively tested. Both in contemporary direct searches for new physics and indirect probes via e.g. flavour anomalies and precise electroweak parameter measurements in proton-electron collisions the SM and as said, it's been consistent with most to date.

% REPEATED ! Given this it is fair to say that the joint description of the  electromagnetic and the  weak  interaction  by  a  single  theory  certainly  is  one  of  major  achievements  of the physical science in this century. 

% REPEATED ? However, the SM  is far from perfect with several open questions that are yet to be fully understood, it is these questions that modern physicists use to justify the research made in the area of high energy physics and Phenomenology. % As a example, one of such weaknesses is a missing explanation of tiny neutrino masses confirmed by flavour-oscillation experiments. 
%
% MOVE BLSM - Given it's successes researchers have long been tempted to try to complete the SM somehow rather than fundamentally alter it. In fact several mechanisms have been proposed that build upon the SM rather than replace it.

The path to the formulation of the SM came from previous principles relating to symmetries in nature, specificity symmetry in physical laws. In fact much in modern physics can be attributed to Emmy Noether. Who deduced trough her first theorem that if the action in a system is invariant under some group of transformations (symmetry), then there exist one or more conserved quantities (constants of motion) which are associated to these transformations. 

This led to the fundamental question behind the SM. Is it possible that upon imposing to a given Lagrangian the invariance under a certain group of symmetries to reach a given form of the dynamics. These dynamics would be particle interactions and this train of thought led to Quantum Electrodynamics (QED).

We can quote Salam and Ward: % A. Salam and J. C. Ward, Nuovo Cim.19, 165 (1961). 

\textit{“Our basic postulate is that it should be possible to generate strong,  weak and electromagnetic  interaction terms (with all their correct symmetry properties and also with clues regarding their relative strengths) by making local gauge transformations on the kinetic energy terms in the free Lagrangian for all particles.”}

We are glossing over a lot of complexity here, and for the SM to be truly new concepts had to be introduced. In the case of weak interactions the presence of very heavy weak gauge bosons require the new concept of spontaneous breakdown of the gauge symmetry and  the Higgs  mechanism. 
% [63,  64,  65].  
% [63]  P. W. Higgs, Phys. Lett.12, 132 (1964).
% [64]  F. Englert and R. Brout, Phys. Rev. Lett.13, 321 (1964).
% [65] G. S. Guralnik, C. R. Hagen, and T. W. B. Kibble, Phys. Rev. Lett.13, 585 (1964).
%
Also, the concept of asymptotic freedom played a crucial role to describe perturbatively the strong interaction at short distances.  
% 
% [89, 90]
% [89]  D. J. Gross and F. Wilczek, Phys. Rev. Lett.30, 1343 (1973).
% [90]  H. D. Politzer, Phys. Rev. Lett.30, 1346 (1973). 

\subsection{Internal symmetry and components of the Standard Model}

The Standard Model spectra after the process of Spontaneous Symmetry breaking (SBB) is composed by, first, the weak force carriers, gauge bosons $W^\pm$ and $Z$, and the photon $\gamma$, the electromagnetic interaction messenger and finally the strong force mediators, the gluons, $g$, as well, of course, by the matter particles, the quarks and leptons. 

Fermions and quarks are organized in three generations each, with 2 pairs by each generation leading to 6 different particles for each family. For quarks we have the up and down for the first generation, charm and strange for the second as well as top and bottom for the third one. Similarly, there are 6 types of leptons, the charged ones, electron, muon and tau, and the associated neutrinos. These are represented in different manners, being that the quarks are represented by the letters $(u,d,c,s,t,b)$ while leptons as $(e,\nu_{e},\mu,\nu_{\mu},\tau,\nu_{\tau})$. 
%
%These physical states that are often denoted as the building blocks of nature. 

However we have not yet explained how such states have acquired their masses and quantum numbers, such as colour and electric charge. To show this, we start by presenting the symmetry group the SM originates from,

\begin{equation}
\mathrm{SU}(3)_{\mathrm{c}} \times \SU{L} \times \U{Y} \quad  .
\label{eq:SM Group}
\end{equation} 

Here we have see the $\mathrm{SU}(3)_{\mathrm{c}}$ group corresponding to quantum chromodynamics (QCD) responsible for the strong force. We'll see this group remains unbroken, while on the other hand we have the $\SU{L} \times \U{Y}$ group that will be broken by the Higgs mechanism into $\U{Q}$. Each particle stems from a field that is charged in a particular manner on each of these groups, making the charge triplets we will come to later define.   

Fermions are half integer spin particles most of which have electrical charge (except the neutrinos).  While quarks interact via the weak, electromagnetic and strong forces, the charged leptons only feel the electromagnetic and weak forces and the neutrinos are weakly interacting.  

A physical fermion is composed of a left-handed and a right-handed field. While the left transform as $SU(2)_L$ doublets and can be written as,

\begin{equation}
L^i= \begin{pmatrix}
\nu_{e_L} \\ e_L 
\end{pmatrix},
\begin{pmatrix}
\nu_{\mu_L} \\ \mu_L 
\end{pmatrix},
\begin{pmatrix}
\nu_{\tau_L} \\ \tau_L 
\end{pmatrix} 
\quad 
\text{and} \quad Q^i= \begin{pmatrix}
u_{L} \\
d_L 
\end{pmatrix},\begin{pmatrix}
c_{L} \\
s_L 
\end{pmatrix}
,\begin{pmatrix}
t_{L} \\
b_L 
\end{pmatrix} \quad ,
\end{equation}
where the $i$ index stands for generation, the latter are $\SU{L}$ singlets and can be simply represented as
%
 \begin{equation}
e^i_R=\{e_R,\mu_R,\tau_R\}, \quad  u^i_R=\{u_R,c_R,t_R\}, \quad d^i_R=\{d_{e_R},s_{e_R},b_{e_R}\} \quad , 
\end{equation}
%
note also that the quarks form triplets of $SU(3)_C$ whereas leptons are colour singlets. The Higgs boson also emerges from an $SU(2)_L$ doublet with the form,
%
\begin{equation}
H=\begin{pmatrix}
\phi^1 + \; i \; \phi^2 \\
\phi^3 + \; i \; \phi^4  
\end{pmatrix} \quad , 
\end{equation}
%
%
%The Lagrangian that describes all vector particles and gauge fields in the SM can be writen as
%
\subsection{Fields and Lagrangian}

The full set of gauge quantum numbers in the SMs original fields is given in tables \ref{table1} and \ref{table2}. 
%
\begin{table}[H]
\centering
\caption{Gauge bosons and Scalar fields in the SM}
\label{table1}
\begin{tabular}{@{}cccccc@{}}
  \hline	
 Fields & Spin 0 field & Spin 1 Field & $SU(3)_C \times SU(2)_L \times U(1)_Y$  \\
  \hline	
 Gluons  & $\times$  & $g$ & (8,1,0) \\	
A bosons & $\times$  & $A^i$ & (1,3,0)   \\
B bosons & $\times$  & $B$ & (1,1,0)   \\
Higgs field & ($\phi^\pm, \phi^0 )$  & $\times$ & (1,2,1) \\ \hline
\end{tabular}
\end{table}
%~
\begin{table}[H]
\centering
\caption{Fermion field dimensions in the SM}
\label{table2}
\begin{tabular}{@{}cccccc@{}}
  \hline	
 Fields & Spin $\frac{1}{2}$ Field & $SU(3)_C \times SU(2)_L \times U(1)_Y$  \\
  \hline	
Quarks (3 gen.) & $Q=(u_L,d_L)$ & $(3,2,\frac{1}{3})$ \\	
$\quad$        & $u_R$ & $(3,1,\frac{4}{3})$   \\
$\quad$   & $d_R$ & $(3,1, -\frac{2}{3})$   \\
Leptons (3 gen.) & $(L=(\nu_{e_L}, e_L )$ & $(1,2,-1)$  \\
$\quad$   & $e_R$ & $(1,1,-2)   $ \\ \hline

\end{tabular}
\end{table}
%
The color charge, weak isospin number and the hypercharge are given by their ordered entries in each triplet (seen in the last column of the tables \ref{table1} and \ref{table2}). Given this content we can then write a Lagrangian invariant under transformations of the $SU(3) \times SU(2) \times U(1)$ as.
%
\begin{align}
\mathcal{L}_{SM} \quad = \quad & (D_\mu H)^\dagger (D^\mu H) - V (H H^\dagger) -  \frac{1}{4} F^i_{\mu \nu} F^{i , \mu \nu} - \frac{1}{4} B_{\mu \nu} B^{\mu \nu} \nonumber \\ 
& + \overline{L_L^i} (i \gamma^\mu D_\mu)  L_L^i +  \overline{Q^i_L} (i \gamma^\mu D_\mu)  Q^i_L +  \overline{L_R^i} (i \gamma^\mu D_\mu)  L_R^i +  \overline{Q^i_R} (i \gamma^\mu D_\mu)  Q^i_R \label{SMfullL}    \\  
 & - [y^d_{jk}\overline{Q}^j_{L} d^k_{R} H +  y_{jk}^u \overline{Q}^j_{L} u^k_{R} \tilde{H} + y^e_{jk} \overline{L}^j e^k_{R} H  + h.c. ] \nonumber \quad , 
\end{align}
%
where $\tilde{H}=i\sigma_2 H$. At the Lagrangian level, gauge interactions are introduced by the gauge covariant
derivative, $D_\mu$, as, 
%
\begin{equation}
\label{eq:PartialDefSM}
D_\mu = \partial_\mu - i g_S \tau^a G^a_\mu - i g T^i A^i_\mu - i g' Y B_\mu \quad ,  
\end{equation}
%
where $\tau^a= \frac{\lambda_a}{2}$ , $(a = 1, . . . , 8)$ are the generators of $SU (3)_c$, $T_i= \frac{\sigma_i}{2} $, $(i = 1, 2, 3)$ are the generators of $SU(2)_L$ and Y is the generator of $U(1)_Y$  Here the symbols $\lambda_a$ and $\sigma_i$ represent the Gell-Mann and Pauli matrices respectively (see appendix). {\color{blue} Add later if there is space}.

In the first line of Eq.\,(\ref{SMfullL}), the first term represents the interactions of gauge bosons with the Higgs field and the second term is the scalar potential associated to the said field. 
%
The second line represents the gauge-kinetic terms and gauge boson self interactions. The third line describes the fermion kinetic terms as well as the interactions among fermions and gauge bosons. Finally, the last line shows the Yukawa interactions between the Higgs and the fermions. It is due to the Yukawa interactions that the SM fermions acquire their masses once the electro-weak (EW) symmetry is broken, as we will later see.

From Eq.\,\ref{eq:PartialDefSM} we will present how the Generators $A^i_\mu$ and $B_\mu$ give rise to the weakly interacting vector bosons $W^\pm$ and $Z^0$ and the electromagnetic vector boson $\gamma$. Contrary to the color sector, where the eight generators $G^a_\mu$ simply correspond to eight gluons $g$ a mediating strong interactions. 

As previously mentioned it is trough the definition of this covariant derivate that we not only impose the invariance under the gauge transformations, but also impose the interaction of charged fields trough the respective gauge bosons. 

\subsection{The Higgs mechanism and the mass generation of the Gauge bosons}

%Of the Gauge group we just defined we will spawn 4 gauge fields named $A_\mu^i$ and $B_\mu$ corresponding respectively to the generators $I^i$ and $Y$. Through observations it was shown that these interact in a very short range requiring them to be massive vector bosons to mathematically describe the proper behaviour. 

%The solution that lead to the attribution of mass to these bosons came trough the mechanism of spontaneous symmetry breaking applied to the Higgs field, and allowed for these 4 fields to be identified as the $W^\pm$ and $Z$ bosons after mixing with goldstones created by the broken symmetries, since the one boson must remains massless, the photon $A^\mu$,we know that only 3 symmetries must be broken, given this the minimal choice for the Higgs field would be a complex scalar doublet $\phi$ represented as 
%\begin{equation}
%\phi = \left( \begin{matrix}
%\phi^+ \\
%\phi^0 
%\end{matrix} \right)
%\end{equation}
%Where the "top"? part of the field is dedicated the to elements with charge while the lower part is neutral. 

%This field has weak isospin charge $I$ of $\frac{1}{2}$ and a hypercharge value of of $1$. This choise allows the breaking of the $SU(2)_L$ group and the $U(1)_Y$ group. The gauge sector in the SM is given by the Lagrangian

From what was defined above, we can now study the SSB of ${\SU{L}\times\U{Y} \rightarrow \U{EW}}$ trough the Higgs Mechanism. Enabling us to find the real physical states of the gauge bosons and the origin of their mass. Let us then consider the part of the Lagrangian containing the scalar covariant derivatives as defined in eq.\,(\ref{SMfullL}) , the scalar potential and the gauge-kinetic terms:
%
\begin{equation}
\mathcal{L}_{gauge} = (D_\mu H)^\dagger (D_\mu H) - V (H H^\dagger) - \frac{1}{4} F^i_{\mu \nu} F^{i , \mu \nu} - \frac{1}{4} B_{\mu \nu} B^{\mu \nu} \quad , 
\label{iamgarbagemanyes}
\end{equation} 
The elements of this sector are defined as,
\begin{equation}
V(H H^\dagger ) = \mu^2 H^\dagger H + \lambda (H^\dagger H)^2 \quad , 
\end{equation}
\begin{equation}
 F^i_{\mu \nu}= \partial_\mu A^i_\nu - \partial_\nu A^i_\mu + g \epsilon_{ijk} A^j_\mu A^k_\nu  \quad , 
\end{equation}
%
In this formulation the constants $g$ and $g^\prime$ are the gauge couplings of the groups $SU(2)_L$ and $U(1)_Y$. 
% 
We expect a phase shift to occur, namely one that ensures $\mu^2 < 0$ and the VEV we are expected to find then takes the form of,
%
\begin{equation}
(H^\dagger H) = \frac{-\mu^2}{2\lambda} = \frac{1}{2} v  \quad , 
\end{equation} 
The VEV is experimentally messured to be $v \approx 246$ GeV. 
%
The vacuum can be aligned in such a way that we have,
\begin{equation}
H_{min} = \frac{1}{\sqrt{2}} \begin{pmatrix} 0 \\
v 
\end{pmatrix} \quad .
\end{equation}

This vacuum will break the $SU(2)_L \times U(1)_Y$ symmetry down to $U(1)_Q$. this means that in the beginning there are four generators, which are $T^{1,2,3}$ and $Y$, and after the breaking we are solely left with one unbroken combination that is $Q =  (T^3 + 1/2)$. This means that in total we will have three broken generators, thus, from Goldstone Theorem, there will be three massless particles. 

As we have seen for the abelian Higgs model, the Goldstones modes can be parametrized as phases in the field space and then can be "rotated away" in the physical basis, leaving us with a single physical massive scalar, the Higgs boson. Note that, with this transformation we are removing three scalar degrees of freedom.  However, they cannot just disappear from the theory and will be absorbed by the massive gauge bosons.
%
In fact, a massless gauge boson contains only two scalar degrees of freedom (transverse polarization). Meanwhile, a massive vector boson has two transverse and a longitudinal polarization, i.e., three scalar degrees od freedom. So, as we discussed above, while before the breaking of the EW symmetry we have four massless gauge bosons, after the breaking we are left with three massive ones. This means that there are three extra scalar degrees of freedom showing up in the gauge sector. It is then commonly said that the goldstone bosons are ``eaten" by the massive gauge bosons and the total number of scalar degrees of freedom in the theory is preserved. Therefore, without loss of generality, we can rewrite the Higgs doublet as

\begin{equation}
 \begin{pmatrix}
G_1 + i G_2 \\ 
v + h(x) + i G_3 
\end{pmatrix} = H (x) \rightarrow H (x) =  \frac{1}{\sqrt{2}} \begin{pmatrix}
0 \\ 
v + h(x) 
\end{pmatrix} \quad .
\label{shame}
\end{equation}
Once the Higgs doublet acquires a VEV, the Lagrangian (\ref{iamgarbagemanyes}) can be recast as:
\begin{align}
\mathcal{L}^\prime = & \frac{1}{2} \partial_\mu h \partial^\mu h - \frac{1}{2} (2v^2 \lambda) h^2
 - \frac{1}{4} F^i_{\mu \nu} F^{i , \mu \nu} - \frac{1}{4} B_{\mu \nu} B^{\mu \nu}  \nonumber \\
& + \frac{1}{8} v^2 g^2 (A^1_\mu A^{1,\mu}+ A^2_\mu A^{2,\mu}) +  \frac{1}{8} v^2  (g^2  A^3_\mu A^{3,\mu} + g^{\prime 2} B_\mu B^\mu - 2 g^2 g^{\prime 2} A^3_\mu B^\mu ) \quad , 
\label{complicatedpart}
\end{align}
A few things become obvious first, we have a lot of mass terms most stemming from the squared gauge fields and a lonesome squared mass term belonging to the real scalar field we know to be the Higgs field. This makes the Higgs boson mass in the SM to be given by,
%
\begin{equation}
M_h= (2v^2 \lambda) \quad .  
\end{equation}
%
To obtain masses for the gauge bosons we need to rotate the gauge fields to a basis where the mass terms are diagonal. First, it is straightforward to see that the electrically charged eigenstates are given by %\ref{gagestate}
%First the fields that carry defined charge that can be easily shown to be 
\begin{equation}
W^\pm_\mu = \frac{1}{\sqrt{2}} (A^{(1)}_\mu \pm i A^{(2)}_\mu) \quad , 
\label{gagestate}
\end{equation}
meaning that the mass of the W bosons is, 
\begin{equation}
M_{W^\pm}= \frac{1}{2} v g \quad .
\end{equation}
The situation becomes a bit more complicated for the second term in (\ref{complicatedpart}) due to a mixing between $A_\mu^3$ and $B_\mu$. In the gauge eigenbasis the mass terms read
\begin{equation}
\begin{pmatrix}
A_\mu^3 && B_\mu
\end{pmatrix} \cdot  \frac{1}{4} \nu ^2 \begin{pmatrix}
g^2  & -g g^\prime \\
-g g^\prime & g^{\prime 2} 
\end{pmatrix} \cdot \begin{pmatrix}
A_\mu^3 \\  B_\mu
\end{pmatrix}  \quad , 
\end{equation} 
which can be diagonalized to obtain
\begin{equation}
\begin{pmatrix}
A_\mu && Z_\mu 
\end{pmatrix} \begin{pmatrix}
0  & 0 \\
0  & \frac{1}{2} v \sqrt{g^2 + g^{\prime 2}} 
\end{pmatrix}  \begin{pmatrix}
A^\mu \\ Z^\mu
\end{pmatrix}  \quad , 
\end{equation}
%Where the new eigenvectors that represent the $Z$ boson and the photon, $A^\mu$ in terms of the former base are written as,
%
we identify the eigenvector associated to the eigenvalue 0 to the photon and the massive one, $ M_Z =  \frac{1}{2} v \sqrt{g^2 + g^{\prime 2}} $, to the Z boson. Such eigenvectors can be written as
%
\begin{align}
A_\mu &=\cos(\theta_\omega) B_\mu + \sin(\theta_\omega) A_\mu^3 \quad ,  \\  
Z_\mu & =- \sin(\theta_\omega) B_\mu + \cos(\theta_\omega) A_\mu^3 \quad , 
\end{align}
%
where $\theta_w$ is the so called Weinberg mixing angle and is defined as
\begin{equation}
\cos(\theta_\omega)=\frac{g}{ \sqrt{g^2 + g^{\prime 2}}} \quad , 
\end{equation}
thus clearly showing the massless photon along with a massive Z boson with mass $M_Z= \frac{1}{2} \nu \sqrt{g^2 + g^{\prime 2}} $. 
%
So we conclude our exploration of the electroweak sector with all the correct massive spectrum observed and its origin discussed.


\subsection{The fermion sector on the Standard Model}

In order to generate mass for the fermions we can have a closer look at the last line in eq. (\ref{SMfullL}). If we replace the Higgs by the shift in Eq. (\ref{shame}) we get, 
%
\begin{align}
\mathcal{L}_y = \, \, & y^d 
\begin{pmatrix}
\overline{u_L} & \overline{d_L} 
\end{pmatrix} d_R 
\begin{pmatrix}
0\\ v + h(x)
\end{pmatrix}  + 
y^s 
\begin{pmatrix}
\overline{c_L} & \overline{s_L} 
\end{pmatrix} s_R 
\begin{pmatrix}
0 \\ v + h(x)
\end{pmatrix} \nonumber  \\ & +
 y^b 
\begin{pmatrix}
\overline{t_L} & \overline{b_L} 
\end{pmatrix} b_R 
\begin{pmatrix}
0 \\ v + h(x)
\end{pmatrix}  +
 y^u 
\begin{pmatrix}
\overline{u_L} & \overline{d_L} 
\end{pmatrix} d_R 
\begin{pmatrix}
v + h(x) \\ 0
\end{pmatrix} \nonumber \\ + &   y^c 
\begin{pmatrix}
\overline{c_L} & \overline{s_L} 
\end{pmatrix} d_R 
\begin{pmatrix}
v + h(x) \\ 0 
\end{pmatrix} +
y^t
\begin{pmatrix}
\overline{t_L} & \overline{b_L} 
\end{pmatrix} t_R \begin{pmatrix}
 v + h(x) \\ 0
\end{pmatrix}   \\ + &  y^e 
\begin{pmatrix}
\overline{\nu_{e_L}} & \overline{e_L} 
\end{pmatrix} e_R 
\begin{pmatrix}
0 \\ v + h(x)
\end{pmatrix} +
 y^\mu 
\begin{pmatrix}
\overline{\nu_{\mu_L}} & \overline{\mu_L} 
\end{pmatrix} \mu_R
\begin{pmatrix}
0 \\ v + h(x)
\end{pmatrix} \nonumber  \\  + & 
 y^\tau \
\begin{pmatrix}
\overline{\nu_{\tau_L}} & \overline{\tau_L} 
\end{pmatrix} \tau_R 
\begin{pmatrix}
0 \\ v + h(x)
\end{pmatrix} +  h.c  \nonumber \quad , 
\end{align}
%
Further expansion of these terms would result in terms like \textit{e.g.} in the electron's case,  
%
\begin{equation}
\mathcal{L}_{y_e} = y^e \; v \left( \overline{e_L} e_R + \overline{e_R} e_L \right) +  y^e h(x) \left( \overline{e_L} e_R + \overline{e_R} e_L \right) \quad , 
\label{electron mass}
\end{equation}
%
where since the electron field is written as, 
% 
\begin{equation}
e=\begin{pmatrix}
e_L \\
e_R 
\end{pmatrix} \quad , 
\end{equation}
%
meaning the first terms in (\ref{electron mass}) equate to electron mass terms as, $m_e \overline{e} e $ and the second terms represent interaction between the electron and the Higgs boson.  
%
This is how the Higgs mechanism generates the mass to all of the fermionic sector except neutrinos due to the SM not containing right handed neutrinos. The absence of neutrino masses contradicts experimental observations. Note, that each field shares the same conserved charges and can be grouped in families with increasing masses their masses depend directly on the term hierarchy of the Yukawa terms. Terms of similar shape are generated for quarks as well, in fact for the example of the up and down quarks we can write. 
%
% PERGUNTAR Note also that the mass term for every lepton depends directly on each of the respective and different Yukawa term and that the SM is a theory where the Yukawa matrices are diagonal by nature not having inter generational terms. {\color{blue} Is it doe? what about FCNCs?}
%
\begin{equation}
M^u_{ij} = Y^u_{ij} \frac{v}{\sqrt{2}} \quad M^d_{ij} = Y^d_{ij} \frac{v}{\sqrt{2}}
\end{equation}
%
where $M$ are the mass matrices for the up-type and down-type quarks have been expressed in terms of generic $3 \times 3$ complex Yukawa matrices and the $v$ the VEV of the Higgs doublet. It is interesting to consider however that these matrices although trivial to diagonalize are not unique, there for the Yukawa are not uniquely defined and a conclusive fundamental explanation of their origin still does not exist.

As a example it would be simple to find a diagonal yukawa matrix to account for the proper mass terms for the quarks. However including non diagonal terms would lead to tree level FCNCs. Which should be present only present at higher orders of perturbation theory, they must be naturally suppressed at tree-level. 

However due to precise quark measurements we can present some constraints on the nature of the Yukawa matrix. Due to the fact that the SM is made to account for the existence of charge parity $\mathcal{CP}$ violation in the form the Cabibbo–Kobayashi–Maskawa (CKM) matrix. 
%
Note, although the interactions of quarks are measured with very high precision, a substantial justification for its hierarchical structure still remains unclear. 
%
The CKM matrix has its origin in charged currents, these currents lead to patterns when diagonalizing mass matrices and rotating quark spinors to the mass basis. 
%
The diagonalization of the real quark states can be performed while leaving space for a generic $3\times3$ matrix, $\mathrm{Y}$ if performed by a set of four bi-unitary matrices, as seen,
\begin{align}
m^u_{diag.}  =  & \frac{v}{\sqrt{2}} \mathrm{U}^u_L \mathrm{Y} {\mathrm{U}^u_R}^\dagger  \\
m^d_{diag.}  = & \frac{v}{\sqrt{2}}  \mathrm{U}^d_L \mathrm{Y} {\mathrm{U}^d_R}^\dagger 
\end{align}

{\color{red} I am having a real hard time explaining the CKM part only affecting the left handed and doing the entire demonstrations I need help}
  