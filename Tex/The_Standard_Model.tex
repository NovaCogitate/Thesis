\chapter{The Standard Model}
\label{ch:SM}

\section{Introduction}

It is hard to question that the Standard Model (SM) is a successful approximately framework whit whom to describe the phenomenology of Particle Physics up to the largest energy scales probed by collider measurements so far. In fact, contemporary direct searches for new physics or indirect probes via e.g. flavour anomalies, have been showing an increasingly consistency with SM predictions. { \color{blue} This is a fundamental study and we'll discuss flavour mechanism in great detail later as many interesting features stem from this. } 

However, it is not untrue to state that the SM also possesses its long list of Achilles heel's, aliments and dubious results leading to serious problems for physicists, this combined with several open questions that are yet to be fully understood justifies the research made in the area of high energy physics and Phenomenology. One of such weaknesses is a missing explanation of tiny neutrino masses confirmed by flavour-oscillation experiments { \color{blue} which we will try to approach later in this dissertation }.  

Due to it's successes researchers have long been tempted to try to complete the SM somehow rather than erase it. In fact several mechanisms have been proposed that build upon the SM rather than replace it. { \color{blue}  We'll investigate some of these in this project. } 

% Retirado do meu Projecto tenho que rescrever  
% Building blocks and theoritical formulation 
\subsection{Composition of the Standard Model}

{ \color{red}
The Standard Model is composed by force carriers, the weak gauge bosons W and Z, the photon, the electromagnetic interaction messenger and the strong force mediators, the gluons, as well by matter particles, the quarks and leptons. Being that the Higgs boson is responsible for the mass generation mechanism.

Fermions are organized in three generations. Furthermore, there are 6 different types of quarks, up and down for the first generation, charm and strange for the second as well as top and bottom for the third one. Similarly, there are 6 types of leptons, the charged ones, electron, muon and tau, and the associated neutrinos, respectively represented by $(u,d,c,s,t,b)$ while leptons as $(e,\nu_{e},\mu,\nu_{\mu},\tau,\nu_{\tau})$

So far we have described the physical states that are often denoted as the building blocks of nature. However we have not yet explained how such states have acquired their masses and gauge quantum numbers, such as colour and electric charge. To see this, we start by noting the the SM is a gauge theory based on the group.

\begin{equation}
SU(3)_c \times SU(2)_L \times U(1)_Y \quad  .
\label{SMsymmetry}
\end{equation} 

Fermions are half integer spin particles most of which have electrical charge (except the neutrinos).  While quarks interact via the weak, electromagnetic and strong forces, the charged leptons only feel the electromagnetic and weak forces and the neutrinos are solely weakly interacting.  

A physical fermion is composed of a left-handed and a right-handed part. While the former transform as $SU(2)_L$ doublets and can be written as,

\begin{equation}
L^i= \begin{pmatrix}
\nu_{e_L} \\ e_L 
\end{pmatrix},
\begin{pmatrix}
\nu_{\mu_L} \\ \mu_L 
\end{pmatrix},
\begin{pmatrix}
\nu_{\tau_L} \\ \tau_L 
\end{pmatrix} 
\quad 
\text{and} \quad Q^i= \begin{pmatrix}
u_{L} \\
d_L 
\end{pmatrix},\begin{pmatrix}
c_{L} \\
s_L 
\end{pmatrix}
,\begin{pmatrix}
t_{L} \\
b_L 
\end{pmatrix} \quad ,
\end{equation}

}




