\chapter{BLSM}
\label{ch:BLSM}

% 
% Introduction of the BLSM.  
% 
% 

\section{ Introduction and motivation for the Model 1}

% Got this from the abstract in my BLSM paper 

Now having discussed the Standard Model. We can introduce the minimal $U(1)_{B-L}$ extension of the Standard Model (B-L-SM). This is a model trough which we can explain neutrino mass generation via through a see-saw mechanism as well as model that contains two new physical particle states, specifically a new Higgs like boson and a Z\textprime gauge Boson. Both these bosons are given mass trough the spontaneous breaking of the $U(1)_{B-L}$ symmetry. This origin for the mass of the referenced bosons means model is already very heavily constricted due to direct searches in the  Large Hadron Collide (LHC). 

One of the goals of this project was to investigate precisely the phenomenological status of the B-L-SM by confronting the new physics predictions with the LHC and electroweak precision data.   

% Unused 
%
% Taking into account the current bounds from direct LHCsearches, we demonstrate that the prediction for the muon (g−2)μanomaly in the B-L-SM yieldsat most a contribution of approximately 8.9×10−12which represents a tension of 3.28 standarddeviations, with the current 1σuncertainty, by means of aZ′boson if its mass lies in a range of6.3 to 6.5 TeV, within the reach of future LHC runs.  This means that the B-L-SM, with heavy yetallowedZ′boson mass range, in practice does not resolve the tension between the observed anomalyin the muon (g−2)μand the theoretical prediction in the Standard Model.  Such a heavyZ′bosonalso implies that the minimal value for a new Higgs mass is of the order of 400 GeV.
% 

The name of B-L-SM stem from the addition of a unitary symmetry $U(1)_{B-L}$, as mentioned, originating from the promotion of a accidental symmetry present in the SM. Thus the Baryon number (B) minus the Lepton number (L) become a fundamental Abelian symmetry group. As a note this model is easily embedded into higher order symmetry groups like for example the SO(10) group meaning this model can be used for the study of Grand Unified Theories.  

% The presence of three generations of right-handed neutrinos also ensures a framework free of anomalies withtheir mass scale developed once the U(1)B−Lis broken by the VEV of a complex SM-singlet scalar field, simultaneouslygiving mass to the correspondingZ′boson.
  
The cosmological consequences of the B-L-SM formulation are also worth mentioning.  First, the presence of an extended neutrino sector implies the existence of a sterile state that can play a role of Dark Matter candidate 

\subsection{Neutrino masses}

As mentioned briefly during the course of this dissertation the SM suffers from lacking a way to explain the observed neutrino masses by default. The minimal way of addressing this problem is by adding heavy Majorana type neutrinos in order to realise a seesaw mechanism. In this chapter we hope to explain how by perform the addition we could generate light neutrino states and how this addition is justified as part of a larger theory. 

 

% Explaining the seesaw. 
 
\subsection{Electro-Weak searches}

\subsubsection{ Oblique parameter analysis }

\subsubsection{ The $( g - 2 )_\mu $ anomaly }

