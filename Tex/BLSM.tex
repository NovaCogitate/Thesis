%\chapter{BLSM}
%\label{ch:BLSM}

\newpage 

% 
% Introduction of the BLSM.  
% 
% 

\section{B-L-SM Model}

% Got this from the abstract in my BLSM paper 

Now having discussed the Standard Model we can begin to look at what might lie beyond it. Here we introduce the minimal $U(1)_{B-L}$ extension of the Standard Model named, B-L-SM. This is a model trough which we can explain neutrino mass generation via a see-saw mechanism as well as by virtue of the model containing two new physical particle states, specifically a new Higgs like boson $H^\prime$ and a $Z^\prime$ gauge Boson, other small deviations in electro-weak measurements, namely the $(g-2)_\mu$ anomaly. This refers to the discrepancy between the measured anomalous magnetic moment of the muon. We can also address the metastability of the electroweak (EW) vacuum in the SM trough the addition of the new scalar allowing for Higgs stabilization up to the plank scale with a the new Higgs starting from few hundred of GeVs. Last, but not least, the presence of the complex SM-singlet $\chi$ interacting with a Higgs doublet typically enhances the strength of EW phase transition potentially converting it into a strong first-order one. Although not covered in this work this analysis is of utmost importance given that it could provide a way to detect new physics and confirm the model without the need for a larger particle collider. This could be pointed to as future work. 

Both these bosons are given mass trough the spontaneous breaking of the $U(1)_{B-L}$ symmetry that gives it's name to the Model. This group originates from the promotion of a accidental symmetry present in the SM, the Baryon number (B) minus the Lepton number (L) to a fundamental Abelian symmetry group. This origin for the mass of the referenced bosons means model is already very heavily constricted due to direct searches in the Large Hadron Collide (LHC). 

One of the goals of this project was to investigate precisely the phenomenological status of the B-L-SM by confronting the new physics predictions with the LHC and electroweak precision data.   

As a note this model is easily embedded into higher order symmetry groups like for example the $SO(10)$ or  or $E_6$ , giving this model the ability to be used for the study of Grand Unified Theories.  

The presence of three generations of right-handed neutrinos instead of a arbitrary number of neutrinos also ensures a framework free of anomalies with their mass scale developed once the $U(1)_{B-L}$ is broken by the VEV, $x$, of a complex SM-singlet scalar field, $\chi$, simultaneously giving mass to the corresponding $Z^\prime$ boson and $H^\prime$.
  
The cosmological consequences of the B-L-SM formulation are also worth mentioning.  First, the presence of an extended neutrino sector implies the existence of a sterile state that can play a role of Dark Matter candidate. That can be completely sterile if stabilized with a $\mathbb{Z}_2$ parity symmetry. Note that the existence of sterile neutrinos can be used to explain the baryon asymmetry via the leptogenesis mechanism. 

\subsection{Formulating the model}

Essentially, the minimal B-L-SM is a Beyond the Standard Model (BSM) framework containing three new ingredients: 

\begin{itemize}
  \item A new gauge interaction
  \item Three generations of right handed neutrinos  
  \item A complex scalar SM-singlet.
\end{itemize}

The first one is well motivated in various GUT scenarios. However note that, if a family-universal symmetry such as $U(1)_{B-L}$ were introduced without changing the SM fermion content, chiral anomalies, which is a non conservative charged current on some channels, involving the $U(1)_{B-L}$ would be generated. These aren't completely undesired by themselves, since their result would be charge conjugation parity symmetry violation, or CP-symmetry violation, a observed missing feature of the SM, but this inclusion would result in far too much of these phenomena. {  \color{red}  (but how do I justify that there would be too much CP-violation?? is this even correct?) }

Secondly, a new sector of additional three $U(1)_{B-L}$ charged Majorana neutrinos is essential for anomaly cancellation. 

Finally is also required that the SM-like Higgs doublet, $H$, does not carry neither baryon nor lepton number, this way it does not participate in the breaking of $U(1)_{B-L}$. It is then necessary to introduce a new scalar singlet, $\chi$, solely charged under $U(1)_{B-L}$, whose VEV breaks the $B-L$ symmetry at a scale higher than the electro-weak breaking scale. It is also this breaking scale that generates masses for heavy neutrinos. 

The particle content and related charges of the minimal $U(1)_{B-L}$ extension of the SM are shown in the table .

\begin{table}[htb!]
\centering
\begin{tabular}{|c|c|c|c|c|c|c|c|c|}
\hline
  & $q_L$  & $u_R$ & $d_R$ & $l_L$  & $e_R$ & $\nu_R$  &  $H$  & $\chi$  \\ \hline
 $SU(3)_c$& 3 & 3 & 3 & 1 & 1 & 1 & 1  & 1  \\
 $SU(2)_L$& 2  & 1 & 1 & 2 & 1 & 1 & 2  & 1 \\
$U(1)_Y$ & $\frac{1}{6}$ & $\frac{2}{3}$  & -$\frac{1}{3}$  & -$\frac{1}{2}$ & -1 & 0 & $\frac{1}{2}$ & 0 \\
$U(1)_{B-L}$ & $\frac{1}{3}$ & $\frac{1}{3}$ & $\frac{1}{3}$  & -1  & -1 &-1  & 0 & 2  \\ \hline 
\end{tabular}
\caption{Quantum fields and their respective quantum numbers in the minimal B-L-SM extension. The last two lines represent the weak and $B-L$ hypercharges}
\label{Charges}
\end{table} 


Given these, we can write the scalar potential of the Lagrangian as,
\begin{equation}
\label{eq:potential}
V(H,\chi) = \mu_1^2 H^\dagger H + \mu_2^2 \chi^\ast \chi + \lambda_1 (H^\dagger H)^2 + \lambda_2 \left(\chi^\ast \chi\right)^2 + \lambda_3  \chi^\ast \chi H^\dagger H
\end{equation}
%
For the scalar Potential to be bounded from below (BFB)) we deduce the conditions,
\begin{equation}
4 \lambda_1 \lambda_2  -  \lambda_3^2 > 0 \quad , \quad \lambda_1 , \lambda_2>0 
\label{eq:BFB}
\end{equation}
%
In this potential, \ref{eq:potential}, the full components of the scalar fields are given by,
\begin{equation}
H = \frac{1}{\sqrt{2}} 
\begin{pmatrix}
-i \( \omega_1 - i \omega_2 \) \\
v + (h + i z)
\end{pmatrix} \quad \chi = \frac{1}{\sqrt{2}} \( x + \(h^\prime + i z^\prime\) \)
\end{equation}
%
In these equations we can see $h$ and $h^\prime$ representing the radial quantum fluctuations around the minimum of the potential that will constitute the physical degrees of freedom associated to the $H$ and $H^\prime$. There are also four Goldstone directions denoted as $\omega_1$, $\omega_2$, $z$ and $z^\prime$ which are absorbed into longitudinal modes of the $W^\pm$, $Z$ and $Z^\prime$ gauge bosons once spontaneous symmetry breaking (SSB) takes place. After SSB the fields take the form, 
%
\begin{equation}
 \langle H \rangle = \frac{1}{\sqrt{2}} 
\begin{pmatrix}
0 \\
v 
\end{pmatrix}	
\qquad
 \langle  \chi \rangle  = \frac{x}{\sqrt{2}}
\label{eq:vacuum}
\end{equation}
% 
where recall $v$ and $x$ are the associated VEVs to each field. From here solving the tadpole equations in relation to each of the VEVs, we arrive at,
\begin{equation}
	v^2 = \tfrac{-\lambda_2 \mu_1^2 + \tfrac{\lambda_3}{2}\mu_2^2}{\lambda_1 \lambda_2 - \tfrac{1}{4}\lambda_3^2} > 0
	\qquad
	\text{and}
	\qquad
	x^2 = \tfrac{-\lambda_1 \mu_2^2 + \tfrac{\lambda_3}{2}\mu_1^2}{\lambda_1 \lambda_2 - \tfrac{1}{4}\lambda_3^2} > 0 
	\label{eq:extremum}
\end{equation}
which, when simplified with the bound from bellow conditions yield a simpler set of equations,
\begin{equation}
\lambda_2 \mu_1^2 < \tfrac{\lambda_3}{2} \mu_2^2 
\qquad
\text{and}
\qquad
\lambda_1 \mu_2^2 < \tfrac{\lambda_3}{2} \mu_1^2
\label{eq:sols}
\end{equation}
Note that although $\lambda_1$ and $\lambda_2$ must be positive to ensure the correct potential shape initially no such conditions exist for the sign of $\lambda_3$ , $\mu_1$ and $\mu_2$. However observing equation \ref{eq:sols} we can infer the following, 

\begin{table}[htb!]
	\begin{center}
		\begin{tabular}{|ccccc|}
			\hline  
			& $\mu_2^2 > 0$ & $\mu_2^2 > 0$ & $\mu_2^2 < 0$ & $\mu_2^2 < 0$  	\\
			& $\mu_1^2 > 0$ & $\mu_1^2 < 0$ & $\mu_1^2 > 0$ & $\mu_1^2 < 0$  	\\        
			\hline
			$\lambda_3 < 0 $     			    							& \xmark		& \checkmark	&	\checkmark & \checkmark	\\
			$\lambda_3 > 0$     			    							& \xmark		& \xmark	&	\xmark &  \checkmark \\
			\hline
		\end{tabular} 
		\caption{Possible Signs of the potential parameters in (\ref{eq:potential}). 
While the \checkmark\,symbol indicates the existence of solutions for tadpole conditions \eqref{eq:sols}, the \xmark\,indicates unstable configurations.}
		\label{tab:signs}  
	\end{center}
\end{table} 
For our studies we decided to leave the sign of $\lambda_3$ unconstrained, choosing a configuration where both $\mu$ parameters are negative. 

Taking the Hessian matrix evaluated at the vaccum value, 
\begin{equation}
\bm{M}^2 =
\begin{pmatrix}
4 \lambda_2 x^2 & \lambda_3 v x \\ 
\lambda_3 v x   & 4 \lambda_1 v^2 
\end{pmatrix}\,,
\label{eq:hess}
\end{equation}



\subsubsection{Neutrino masses}

As mentioned briefly during the course of this dissertation the SM suffers from lacking a way to explain the observed neutrino masses by default. The minimal way of addressing this problem is by adding heavy Majorana type neutrinos in order to realise a seesaw mechanism. In this chapter we hope to explain how by perform the addition we could generate light neutrino states and how this addition is justified as part of a larger theory. 

% Explaining the seesaw. 
 
\subsection{Electro-Weak searches}

\subsubsection{ Oblique parameter analysis }

\subsubsection{ The $( g - 2 )_\mu $ anomaly }

