%\chapter{BLSM}
%\label{ch:BLSM}

\newpage 

% 
% Introduction of the BLSM.  
% 
% 

\section{B-L-SM Model}

% Got this from the abstract in my BLSM paper 

Having discussed the Standard Model, we are ready to look at what might lie beyond it. In this chapter we introduce the minimal $U(1)_{B-L}$ extension of the Standard Model named, B-L-SM. This is a model trough which we can explain neutrino mass generation via a simple see-saw mechanism as well as, by virtue of the model containing two new physical particle states, specifically a new Higgs like boson $H^\prime$ and a $Z^\prime$ gauge Boson, other small deviations in electro-weak measurements, namely the $(g-2)_\mu$ anomaly. This refers to the discrepancy between the measured anomalous magnetic moment of the muon. 

Both the additional bosons are given mass mostly trough the spontaneous breaking of the $U(1)_{B-L}$ symmetry that gives it's name to the Model. This group originates from the promotion of a accidental symmetry present in the SM, the Baryon number (B) minus the Lepton number (L) to a fundamental Abelian symmetry group. This origin for the mass of the referenced bosons means model is already very heavily constricted due to long-standing direct searches in the Large Hadron Collide (LHC). 

Trough this model we can also address the metastability of the electroweak (EW) vacuum in the SM trough the addition of the new scalar. Allowing for Higgs stabilization up to the plank scale with a the new Higgs starting from few hundred of GeVs. 

Lastly, the presence of the complex SM-singlet $\chi$ interacting with a Higgs doublet typically enhances the strength of EW phase transition potentially converting it into a strong first-order one. Although not covered in this work this analysis is of utmost importance given that it could provide a way to detect new physics and confirm the model without the need for a larger particle collider. This could be pointed to as future work. { \color{red} Verify if the BLSM can be seen in LISA or stuff like that. This part needs new citations }

One of the goals of this project was to investigate precisely the phenomenological status of the B-L-SM by confronting the new physics predictions with the LHC and electroweak precision data.   

As a note this model is easily embedded into higher order symmetry groups like for example the $SO(10)$ or  or $E_6$ , giving this model the ability to be used for the study of Grand Unified Theories.  

The presence of three generations of right-handed neutrinos instead of a arbitrary number of neutrinos also ensures a framework free of anomalies with their mass scale developed once the $U(1)_{B-L}$ is broken by the VEV, $x$, of a complex SM-singlet scalar field, $\chi$, simultaneously giving mass to the corresponding $Z^\prime$ boson and $H^\prime$.
  
The cosmological consequences of the B-L-SM formulation are also worth mentioning.  First, the presence of an extended neutrino sector implies the existence of a sterile state that can play the role of Dark Matter candidate. These can be completely sterile if stabilized with a $\mathbb{Z}_2$ parity symmetry. Note that the existence of sterile neutrinos can be used to explain the baryon asymmetry via the leptogenesis mechanism. 

\subsection{Formulating the model}

Essentially, the minimal B-L-SM is a Beyond the Standard Model (BSM) framework containing three new ingredients: 

\begin{itemize}
  \item A new gauge interaction
  \item Three generations of right handed neutrinos  
  \item A complex scalar SM-singlet.
\end{itemize}

The first one is well motivated in various GUT scenarios. However note that, if a family-universal symmetry such as $U(1)_{B-L}$ were introduced without changing the SM fermion content, chiral anomalies, which is a non conservative charged current on some channels, involving the $U(1)_{B-L}$ would be generated. These aren't completely undesired by themselves, since their result would be charge conjugation parity symmetry violation, or CP-symmetry violation, a observed missing feature of the SM, but this inclusion would result in far too much of these phenomena. {  \color{red}  (but how do I justify that there would be too much CP-violation?? is this even correct?) }

Secondly, a new sector of additional three $U(1)_{B-L}$ charged Majorana neutrinos is essential for anomaly cancellation. 

Finally, the SM-like Higgs doublet, $H$, does not carry neither baryon nor lepton number, this way it does not participate in the breaking of $U(1)_{B-L}$. It is then necessary to introduce a new scalar singlet, $\chi$, solely charged under $U(1)_{B-L}$, whose VEV breaks the $B-L$ symmetry. It is also this breaking scale that generates masses for heavy neutrinos. As mentioned this breaking occurs before the EW breaking.

The particle content and related charges of the minimal $U(1)_{B-L}$ extension of the SM are shown in the table. Note these are similar to the SM as to be expected. 

\begin{table}[htb!]
\centering
\begin{tabular}{|c|c|c|c|c|c|c|c|c|}
\hline
  & $q_L$  & $u_R$ & $d_R$ & $l_L$  & $e_R$ & $\nu_R$  &  $H$  & $\chi$  \\ \hline
 $SU(3)_c$& 3 & 3 & 3 & 1 & 1 & 1 & 1  & 1  \\
 $SU(2)_L$& 2  & 1 & 1 & 2 & 1 & 1 & 2  & 1 \\
$U(1)_Y$ & $\frac{1}{6}$ & $\frac{2}{3}$  & -$\frac{1}{3}$  & -$\frac{1}{2}$ & -1 & 0 & $\frac{1}{2}$ & 0 \\
$U(1)_{B-L}$ & $\frac{1}{3}$ & $\frac{1}{3}$ & $\frac{1}{3}$  & -1  & -1 &-1  & 0 & 2  \\ \hline 
\end{tabular}
\caption{Quantum fields and their respective quantum numbers in the minimal B-L-SM extension. The last two lines represent the weak and $B-L$ hypercharges}
\label{tab:charges}
\end{table} 

\subsubsection{Scalar sector}

Given the information we now posses we can begin examining the new Lagrangian terms. Starting by the scalar potential, 
%
\begin{equation}
\label{eq:potential}
V(H,\chi) = \mu_1^2 H^\dagger H + \mu_2^2 \chi^\ast \chi + \lambda_1 (H^\dagger H)^2 + \lambda_2 \left(\chi^\ast \chi\right)^2 + \lambda_3  \chi^\ast \chi H^\dagger H
\end{equation}
%
For the scalar Potential to be bounded from below (BFB). BFB conditions exist fundamentally to ensure there is a single global minima. Studying the potential \ref{eq:potential} we deduce the conditions,
\begin{equation}
4 \lambda_1 \lambda_2  -  \lambda_3^2 > 0 \quad , \quad \lambda_1 , \lambda_2>0 
\label{eq:BFB}
\end{equation}
%
{ \color{blue} Should I explain what is bound from bellow? Basically we must ensure there is a single globla minima.} 
%
Where the full components of the scalar fields are given by,
\begin{equation}
H = \frac{1}{\sqrt{2}} 
\begin{pmatrix}
-i \( \omega_1 - i \omega_2 \) \\
v + (h + i z)
\end{pmatrix} \quad \chi = \frac{1}{\sqrt{2}} \( x + \(h^\prime + i z^\prime\) \)
\end{equation}
%
In these equations we can see $h$ and $h^\prime$ representing the radial quantum fluctuations around the minimum of the potential. These will constitute the physical degrees of freedom associated to the $H$ and $H^\prime$. There are also four Goldstone directions denoted as $\omega_1$, $\omega_2$, $z$ and $z^\prime$ which are absorbed into longitudinal modes of the $W^\pm$, $Z$ and $Z^\prime$ gauge bosons once spontaneous symmetry breaking (SSB) takes place. After SSB the associated VEVs take the form, 
%
\begin{equation}
 \langle H \rangle = \frac{1}{\sqrt{2}} 
\begin{pmatrix}
0 \\
v 
\end{pmatrix}	
\qquad
 \langle  \chi \rangle  = \frac{x}{\sqrt{2}}
\label{eq:vacuum}
\end{equation}
% 
here, recall $v$ and $x$ are the associated VEVs to each field. From here we can solve the tadpole equations in relation to each of the VEVs as to ensure non-zero minima, we arrive at,
%
\begin{equation}
	v^2 = \tfrac{-\lambda_2 \mu_1^2 + \tfrac{\lambda_3}{2}\mu_2^2}{\lambda_1 \lambda_2 - \tfrac{1}{4}\lambda_3^2} > 0
	\qquad
	\text{and}
	\qquad
	x^2 = \tfrac{-\lambda_1 \mu_2^2 + \tfrac{\lambda_3}{2}\mu_1^2}{\lambda_1 \lambda_2 - \tfrac{1}{4}\lambda_3^2} > 0 
	\label{eq:extremum}
\end{equation}
%
which, when simplified with the bound from bellow conditions yield a simpler set of equations,
%
\begin{equation}
\lambda_2 \mu_1^2 < \tfrac{\lambda_3}{2} \mu_2^2 
\qquad
\text{and}
\qquad
\lambda_1 \mu_2^2 < \tfrac{\lambda_3}{2} \mu_1^2
\label{eq:sols}
\end{equation}
%
Note that although $\lambda_1$ and $\lambda_2$ must be positive to ensure the correct potential shape, no such conditions exist for the sign of $\lambda_3$ , $\mu_1$ and $\mu_2$. However observing equation \ref{eq:sols} we can infer that some combinations of signs are impossible, 
%
\begin{table}[htb!]
	\begin{center}
		\begin{tabular}{|ccccc|}
			\hline  
			& $\mu_2^2 > 0$ & $\mu_2^2 > 0$ & $\mu_2^2 < 0$ & $\mu_2^2 < 0$  	\\
			& $\mu_1^2 > 0$ & $\mu_1^2 < 0$ & $\mu_1^2 > 0$ & $\mu_1^2 < 0$  	\\        
			\hline
			$\lambda_3 < 0 $     			    							& \xmark		& \checkmark	&	\checkmark & \checkmark	\\
			$\lambda_3 > 0$     			    							& \xmark		& \xmark	&	\xmark &  \checkmark \\
			\hline
		\end{tabular} 
		\caption{Possible Signs of the potential parameters in (\ref{eq:potential}). 
While the \checkmark\,symbol indicates the existence of solutions for tadpole conditions \eqref{eq:sols}, the \xmark\,indicates unstable configurations.}
		\label{tab:signs}  
	\end{center}
\end{table} 
%
For our studies we decided to leave the sign of $\lambda_3$ positive, choosing a configuration where both $\mu$ parameters are negative. This doesn't directly translate to any real physical consequence.  
%
These conditions now established we proceed to investigate the physical states of B-L-SM scalar sector. By first, taking the Hessian matrix evaluated at the vacuum value, 
%
\begin{equation}
\mathbf{M}^2 =
\begin{pmatrix}
4 \lambda_2 x^2 & \lambda_3 v x \\ 
\lambda_3 v x   & 4 \lambda_1 v^2 
\end{pmatrix}\,,
\label{eq:hess}
\end{equation}
% 
Moving this matrix to it's physical mass eigenbase, we obtain the following eigenvalues,
%
\begin{equation}
m_{h_{1,2}}^2 = \lambda_1 v^2 + \lambda_2 x^2 \mp \sqrt{(\lambda_1 v^2 - \lambda_2 x^2)^2 + (\lambda_3 x v)^2}\,,
\label{eq:eigvals}
\end{equation}
The physical basis vectors $h_1$ and $h_2$ can then be related to the original fields of gauge eigenbasis $h$ and $h^\prime$ trough a simple rotation matrix:
%
\begin{equation}
	\begin{pmatrix}
	h_1 \\
	h_2 
	\end{pmatrix}
	=
	\mathbf{O}
	\begin{pmatrix}
	h \\
	h^\prime 
	\end{pmatrix}\,.
	\label{eq:trans}
\end{equation}
%
The rotation matrix being written as, 
%
\begin{equation}
	\mathbf{O} = 
	\begin{pmatrix}
	\cos \alpha_h & -\sin \alpha_h \\
	\sin \alpha_h & \cos \alpha_h 
	\end{pmatrix}\,.
	\label{eq:rotmat}
\end{equation}
%
Recall that due to the SSB order $x > v$. { \color{blue} And here the mixing angle is represented simply by, 
\begin{equation}
\tan 2 \alpha_h   = \frac{ \left| \lambda_3 \right|  v v^\prime }{\lambda_1 v^2 - \lambda_2 v^{\prime\,^2} } 
\end{equation} }
It is worth presenting the case of approximate decoupling where, $v/x\ll 1$. In this case scalar masses and mixing angle become particularly simple,
\begin{equation}
\sin \alpha_h \approx \dfrac{1}{2}\dfrac{\lambda_3}{\lambda_2} \dfrac{v}{x} \qquad
m_{h_1}^2 \approx 2 \lambda_1 v^2 \qquad m_{h_2}^2 \approx 2 \lambda_2 x^2
\label{eq:simplify}
\end{equation}
Given the mass scale of our results, these equations serve as a good 
approximation for most of the phenomenologically consistent points 
in our numerical analysis discussed below.

\subsubsection{Gauge Sector}

Moving onto the gauge boson and Higgs kinetic terms in the B-L-SM, consider the following portion of the Lagrangian,
\begin{equation}
\mathcal{L}_{U(1)'s} =  \left| D_\mu H \right|^2 + \left| D_\mu \chi \right|^2 -\dfrac{1}{4} F_{\mu \nu} F^{\mu \nu} -\dfrac{1}{4} F^\prime_{\mu \nu} F^{\prime \mu \nu} -\dfrac{1}{2} \kappa F_{\mu \nu} F^{\prime \mu \nu}
\label{eq:Lu1}
\end{equation}
where $F^{\mu \nu}$ and $F^{\prime \mu \nu}$ are the standard field strength tensors, respectively for the hypercharge $U(1)_{Y}$ and B minus L $U(1)_{B-L}$, 
\begin{equation}
	F_{\mu \nu} = \partial_\mu A_\nu - \partial_\nu A_\mu 
	\qquad
	\text{and}
	\qquad
	 F^\prime_{\mu \nu} = \partial_\mu A^\prime_\nu - \partial_\nu A^\prime_\mu\,.
	 \label{eq:Fmn}
\end{equation}
written in terms of the gauge fields $A_\mu$ and $A_\mu^\prime$, respectively. Given this is a model with two Unitary groups we must consider the possible mixing in between these groups. This shall be parametrized trough a $\kappa$ factor.

The Abelian part of the covariant derivative in equation \ref{eq:Lu1} is given by,
\begin{equation}
	D_\mu \supset i g_1 Y A_\mu + i g_1^\prime Y_{\rm B-L} A_\mu^\prime\,,
\end{equation} 
% 
with $g_1$ and $g_1^\prime$ being the $U(1)_{Y}$ and $U(1)_{B-L}$ the gauge couplings with the $Y$ and $B-L$ charges are specified in Tab.~\ref{tab:charges}. However it is convenient to rewrite the gauge kinetic terms in the canonical form, i.e.
%
\begin{equation}
	F_{\mu \nu} F^{\mu \nu} + F^\prime_{\mu \nu} F^{\prime \mu \nu} + 2 \kappa F_{\mu \nu} F^{\prime \mu \nu} \to B_{\mu \nu} B^{\mu \nu} + B^\prime_{\mu \nu} B^{\prime \mu \nu}\,.
	\label{eq:AtoB}
\end{equation}
%
A generic orthogonal transformation in the field space does not eliminate the kinetic mixing term. So, in order to satisfy Eq.~\eqref{eq:AtoB} an extra non-orthogonal transformation should be imposed such that Eq.~\eqref{eq:AtoB} is realized. Taking $\kappa = \sin \alpha$, a suitable redefinition of fields $\{A_\mu,A_\mu^\prime\}$ into $\{B_\mu, B_\mu^\prime\}$ that eliminates $\kappa$-term according to Eq.~\eqref{eq:Lu1} can be cast as
\begin{equation}
	\begin{pmatrix}
	A_\mu \\
	A^\prime_\mu 
	\end{pmatrix}
	=
	\begin{pmatrix}
	1 & -\tan \alpha \\
	0 & \sec \alpha 
	\end{pmatrix}
	\begin{pmatrix}
	B_\mu \\
	B^\prime_\mu 
	\end{pmatrix}\,,
	\label{eq:trans-kappa}
\end{equation}
Note there is a limit without kinitic mixing where $\alpha = 0$. Note that this transformation is generic and valid for any basis in the field space. The transformation (\ref{eq:trans-kappa}) results in a modification of the covariant derivative that acquires two additional terms encoding the details of the kinetic mixing, i.e.

\begin{equation}
D_\mu \supset \partial_\mu + i \(g_Y \; Y + g_BY \; Y_{B-L}\) B_\mu + i \(g_{B-L} \; Y_{B-L} + g_{YB} \; Y\) B_\mu^\prime\,,
\label{eq:newCov}
\end{equation}	
where the gauge couplings take the form
\begin{equation}
	\begin{cases}
	g_Y = g_1 \\
	g_{B-L} = g_1^\prime \sec \alpha \\
	g_{YB} = -g_1 \tan \alpha \\
	g_{BY} = 0
	\end{cases} \,,
	\label{eq:new-g-simp}
\end{equation}
which is the standard convention in the literature. The resulting mixing between the neutral gauge fields including $Z^\prime$ can be represented as follows
%
\begin{equation}
\begin{aligned}
\begin{pmatrix}
\gamma_\mu \\
Z_\mu \\
Z^\prime_\mu
\end{pmatrix}
=
\begin{pmatrix}
\cos \theta_W & \sin \theta_W & 0\\
-\sin \theta_W \cos \theta_W^\prime & \cos \theta_W \cos \theta_W^\prime & \sin \theta_W^\prime \\
\sin \theta_W \sin \theta_W^\prime & -\cos \theta_W^\prime \sin \theta_W^\prime & \cos \theta_W^\prime
\end{pmatrix}
\begin{pmatrix}
B_\mu \\
A^3_\mu \\
B^\prime_\mu
\end{pmatrix}
\end{aligned}
\label{eq:g-Z-Zp}
\end{equation}	
%
where $\theta_W$ is the weak mixing angle and $\theta^\prime_W$ is defined as
\begin{equation}
\sin(2 \theta^\prime_W) = \frac{2 g_{YB} \sqrt{g^2 + g_{Y}^2}}{\sqrt{(g_{YB}^2 + 16 (\frac{x}{v})^2 g_{B-L}^2 - g^2 - g_{Y}^2)^2 + 4 g_{YB}^2 (g^2 + g_{Y}^2)} }\,,
\label{eq:theta-p-full}
\end{equation}
in terms of $g$ and $g_{Y}$ being the $SU(2)_{L}$ and $U_{Y}$ gauge couplings, respectively. In the physically relevant limit, $v/x \ll 1$, the above expression greatly simplifies leading to
\begin{equation}
	\sin \theta_W^\prime \approx \dfrac{1}{16
	} \dfrac{g_{YB}}{g_{B-L}}\( \dfrac{v}{x} \)^2 \sqrt{g^2 + g_{Y}^2} \,,
	\label{eq:theta-p}
\end{equation}
%
up to $(v/x)^3$ corrections. In the limit of no kinetic mixing, i.e. $g_{YB} \to 0$, there is no mixture of $Z^\prime$ and SM gauge bosons. 

Note, the kinetic mixing parameter $\theta_W^\prime$ has rather stringent constraints from $Z$ pole experiments both at the Large Electron-Positron Collider (LEP) and the Stanford Linear Collider (SLC), restricting its value to be smaller than $10^{-3}$ approximately, which we set as an upper bound in our numerical analysis. Expanding the kinetic terms $\left| D_\mu H \right|^2 + \left| D_\mu \chi \right|^2$ around the vacuum one can extract the following mass matrix for vector bosons
\begin{equation}
	m_V^2 =
	\dfrac{v^2}{4}
	\begin{pmatrix}
	g^2 \;\;&\;\; 0 \;\;&\;\; 0 \;\;&\;\; 0 \;\;&\;\; 0 \\
	0 \;\;&\;\; g^2 \;\;&\;\; 0 \;\;&\;\; 0 \;\;&\;\; 0 \\
	0 \;\;&\;\; 0 \;\;&\;\; g^2 \;\;&\;\; -g g_{Y} \;\;&\;\; -g g_{YB} \\
	0 \;\;&\;\; 0 \;\;&\;\; -g g_{Y} \;\;&\;\; g_{Y}^2 \;\;&\;\; g_{Y} g_{YB} \\
	0 \;\;&\;\; 0 \;\;&\;\; -g g_{YB} \;\;&\;\; g_{Y} g_{YB} \;\;&\;\; g_{YB}^2 + 16 \(\dfrac{x}{v}\)^2 g_{B-L}^2
	\end{pmatrix}
\end{equation}
%
whose eigenvalues read
\begin{equation}
	m_A = 0 \, \text{,} \qquad m_W = \tfrac{1}{2} v g
\end{equation}
corresponding to physical photon and $W^\pm$ bosons as well as
\begin{equation}
m_{Z,Z^\prime}=\sqrt{g^2 + g^2_{Y}} \cdot \frac{v}{2}  \sqrt{\frac{1}{2} \left( \frac{g_{YB}^2 + 16 (\frac{x}{v})^2 g^2_{\rm BL} }{g^2 + g^2_{\rm Y}} +1  \right) \mp \frac{g_{YB}}{\sin(2 \theta_W^\prime) \sqrt{g^2 + g^2_{\rm Y}}}}\,.
\label{eq:ZZp-mass}
\end{equation}
for two neutral massive vector bosons, with one of them, not necessarily the lightest, representing the SM-like $Z$ boson. It follows from LEP and SLC constraints on $\theta_W^\prime$, that Eq.~\eqref{eq:theta-p} also implies that either $g_{YB}$ or the ratio $\tfrac{v}{x}$ are small. In this limit, Eq.~\eqref{eq:ZZp-mass} simplifies to
\begin{equation}
	m_Z \approx \tfrac{1}{2} v \sqrt{g^2 + g_{Y}^2} \qquad \text{and} \qquad m_{Z^\prime} \approx 2 g_{B-L} x\,,
	\label{eq:mZ}
\end{equation}
%
where the $m_{Z^\prime}$ depends only on the SM-singlet VEV ,$x$ and on the $U(1)_{B-L}$ gauge coupling and will be attributed to a heavy $Z^\prime$ state, while the light $Z$-boson mass corresponds to its SM value.

\subsection{The Yukawa sector}

One of the key features of the B-L-SM is the presence of non-zero neutrino masses. In its minimal version, such masses are generated via a type-I seesaw mechanism. The Yukawa Lagrangian of the model reads
\begin{equation}
\begin{aligned}
\mathcal{L}_f = 
-Y_u^{ij} \overline{q_{\rm L i}} u_{\rm R j} \widetilde{H} 
-Y_d^{ij} \overline{q_{\rm L i}} d_{\rm R j} H
-Y_e^{ij} \overline{\ell_{\rm L i}} e_{\rm R j} H
- Y_\nu^{ij} \overline{\ell_{\rm L i}} \nu_{\rm R j} \widetilde{H}
	-\dfrac{1}{2} Y_\chi^{ij} \overline{\nu_{\rm R i}^c} \nu_{\rm R j} \chi + {\rm c.c.}
\end{aligned}
\label{eq:Yuk}
\end{equation}
Notice that Majorana neutrino mass terms of the form $M \overline{\nu_{R}^c} \nu_{R}$ would explicitly violate the $U(1)_{B-L}$ symmetry and are therefore not present. In Eq.~\eqref{eq:Yuk}, $Y_u$, $Y_d$ and $Y_e$ are the $3 \times 3$ Yukawa matrices that reproduce the quark and charged lepton sector of the SM, while $Y_\nu$ and $Y_\chi$ are the new Yukawa matrices responsible for the generation of neutrino masses and mixing. In particular, one can write
\begin{equation}
	\mathbf{m}_{\nu_l}^{Type-I} = \dfrac{1}{\sqrt{2}}\dfrac{v^2}{x} \mathbf{Y}_\nu^t \mathbf{Y}^{-1}_\chi \mathbf{Y}_\nu\,,
\end{equation}
%
for light $\nu_l$ neutrino masses, whereas the heavy $\nu_h$ ones are given by
\begin{equation}
	\mathbf{m}_{\nu_h}^{Type-I} \approx \dfrac{1}{\sqrt{2}} \mathbf{Y}_\chi x\,,
\end{equation} 
where we have assumed a flavour diagonal basis. Note that the smallness of light neutrino masses imply that either the $x$ VEV is very large or (if we fix it to be at the $\mathcal{O}\left({TeV}\right)$ scale and $\mathbf{Y}_\chi \sim \mathcal{O}\(1\right)$) the corresponding Yukawa coupling should be tiny, $\mathbf{Y}_\nu < 10^{-6}$. It is clear that the low scale character of the type-I seesaw mechanism in the minimal B-L-SM is \textit{faked} by small Yukawa couplings to the Higgs boson. A more elegant description was proposed in Ref.~\cite{Khalil:2010iu} where small SM neutrino masses naturally result from an inverse seesaw mechanism. In this work, however, we will not study the neutrino sector and thus, for an improved efficiency of our numerical analysis of $Z^\prime$ observables, it will be sufficient to fix the Yukawa couplings to $\mathbf{Y}_\chi = 10^{-1}$ and $\mathbf{Y}_\nu = 10^{-7}$ values such that the three lightest neutrinos lie in the sub-eV domain.

%\subsubsection{Neutrino masses}

%As mentioned briefly during the course of this dissertation the SM suffers from lacking a way to explain the observed neutrino masses by default. The minimal way of addressing this problem is by adding heavy Majorana type neutrinos in order to realise a seesaw mechanism. In this chapter we hope to explain how by perform the addition we could generate light neutrino states and how this addition is justified as part of a larger theory. 

% Explaining the seesaw. 

\subsection{Phenomenological analysis}

The numerical portion of the study that was developed during the course of this thesis had the goal to manufacture a wide scan across all the possible parameter space thus testing all the limits of the model. We required information relating several constraints 

During this thesis I developed a set of automatic tools that verify a large number of constraints. 


 
%\subsection{Electro-Weak searches}

%\subsubsection{ Oblique parameter analysis }

%\subsubsection{ The $( g - 2 )_\mu $ anomaly }

