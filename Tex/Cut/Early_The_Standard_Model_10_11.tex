%\chapter{The Standard Model}
%\label{ch:SM}

\newpage 

\section{The Standard Model of Particle Physics}

To pave the way for our future studies we present a brief overview of the SM. 

\subsection{Introduction}

It is hard to question that the Standard Model (SM) describes successful approximate framework with whom to describe the phenomenology of Particle Physics up to the largest energy scales probed by collider measurements so far. Proposed in the sixties by Glashow, Salam and Weinberg it has been extensively tested and in contemporary direct searches for new physics or indirect probes via e.g. flavour anomalies and precise electroweak parameter measured in proton-electron collisions, has been showing an increasingly consistency with real results.  % { \color{blue} These will show themselves fundamental for our study and we'll discuss these mechanisms in greater detail  over the course of this work. } % flavour mechanism in great detail later as many interesting features stem from this. . 
Given this it is fair to say that the joint description of the  electromagnetic and the  weak  interaction  by  a  single  theory  certainly  is  one  of  major  achievements  of the physical science in this century. 

However, the SM  is far from perfect with several open questions that are yet to be fully understood, it is these questions that modern physicists use to justify the research made in the area of high energy physics and Phenomenology. As a example, one of such weaknesses is a missing explanation of tiny neutrino masses confirmed by flavour-oscillation experiments. 

Given it's successes researchers have long been tempted to try to complete the SM somehow rather than fundamentally alter it. In fact several mechanisms have been proposed that build upon the SM rather than replace it. { \color{blue}  We'll investigate some of these in this project (BLSM 3HDM) . } 

\subsection{Up to Gauge Theory?}

It is well known that symmetry played a very important role in the development of modern physics ever since Emmy Noether's first theorem, which derives conserved quantities from symmetries. Precisely the theorem states if an action is invariant under some group of transformations (symmetry), then there exist one or more conserved quantities (constants of motion) which are associated to these transformations. 

The question that the lead to the framework of the standard model was:  upon imposing to a given Lagrangian the invariance under a certain symmetry, would it be possible to determine the form of the interaction among the particles? In other words, could symmetry also imply dynamics. This train of thought led to Quantum Electrodynamics (QED) the first successful prototype of quantum field theory.

In QED the existence and some of the properties of the gauge field (which we'll later identify as the photon) follow from a principle of invariance under local gauge transformations of the $U(1)$ group.

We can quote Salam and Ward: % A. Salam and J. C. Ward, Nuovo Cim.19, 165 (1961). 

\textit{“Our basic postulate is that it should be possible to generate strong,  weak and electromagnetic  interaction terms (with all their correct symmetry properties and also with clues regarding their relative strengths) by making local gauge transformations on the kinetic energy terms in the free Lagrangian for all particles.”}

We are glossing over a lot of complexity here, for the SM to be truly  complete Noether's theorem alone wouldn't suffice and new concepts had to be introduced. In the case of weak interactions the presence of very heavy weak gauge bosons require the new concept of spontaneous breakdown  of  the  gauge  symmetry and  the Higgs  mechanism. 
% [63,  64,  65].  
% [63]  P. W. Higgs, Phys. Lett.12, 132 (1964).
% [64]  F. Englert and R. Brout, Phys. Rev. Lett.13, 321 (1964).
% [65] G. S. Guralnik, C. R. Hagen, and T. W. B. Kibble, Phys. Rev. Lett.13, 585 (1964).
%
Also, the  concept  of  asymptotic  freedom 
% 
% [89, 90]
% [89]  D. J. Gross and F. Wilczek, Phys. Rev. Lett.30, 1343 (1973).
% [90]  H. D. Politzer, Phys. Rev. Lett.30, 1346 (1973). 
played a crucial role to describe perturbatively the strong interaction at short distances, making the strong gauge bosons trapped. 
% Should I meantion quantum chromodynamics? 

\subsubsection{Symmetries }

% para later convience I should Use dots above phi ... geez 

A symmetry can be very broadly defined as a property of a system that is preserved or remains unchanged. However for our interests we are going to look at field transformations that leave a Lagrangian system invariant. To exemplify this consider the following generic transformation of a field $\phi$:
\begin{equation}
\phi \longrightarrow \phi^\prime = \phi + \delta \phi 
\end{equation} 
To be invariant means the langraingian will be unchanging, thus, 
\begin{equation}
\mathcal{L}(\phi , \frac{d \phi}{dt}  )  = \mathcal{L}(\phi^\prime , \frac{d \phi^\prime }{dt}  )
\end{equation}
Noether explored this relation, noting the Lagrangian would transform itself like, 
\begin{equation}
\mathcal{L}(\phi, \partial \phi) \longrightarrow \mathcal{L^\prime} (\phi+\delta \phi , \partial \phi + \delta \partial \phi )  
\end{equation}
leading to the form,
\begin{equation}
\mathcal{L}^\prime  = \mathcal{L}(\phi , \partial \phi ) 
+ \partial \phi \frac{\partial \mathcal{L}}{\partial\phi} 
+  \delta \frac{d\phi }{dt} \frac{\partial \mathcal{L}}{ \partial \frac{d\phi}{dt} }
\end{equation}
where assuming the equations of motion are satisfied $\left( \frac{ \partial \mathcal{L}}{\partial \phi}  = \frac{d}{dt} \frac{ \partial \mathcal{L}}{\frac{d \phi}{d t}} \right)$ we can reach a expression for the first order change in the Lagrangian given by, 
\begin{equation}
\mathcal{L}^\prime = \mathcal{L} + \frac{d}{dt} \left( \frac{ \partial \mathcal{L}}{ \partial \frac{d \phi}{dt}} \delta \phi \right) 
\end{equation}
Here we define, $j$, as the Noether Current, 
\begin{equation}
j= \frac{ \partial \mathcal{L}}{ \partial \frac{d \phi}{dt}} \delta \phi  
\end{equation}
This way we can define a transformation, $\delta \phi$ that leaves the action invariant, as, 
\begin{equation}
\delta \mathcal{S} = 0 \implies \delta \mathcal{L} = 0 \implies \frac{ \partial \mathcal{L}}{ \partial \frac{d \phi}{dt}} \delta \phi  = 0 \implies \frac{d \, j}{dt} = 0
\end{equation}
This way we can say that $j$ is constant, this means there is a conversed quantity. A simple real example of this would be the case of a projectile in Lagrangian physics. The lagrangian would be,
\begin{equation}
\mathcal{L} = \frac{1}{2}m\left(\frac{d\,x^2}{dt} + \frac{d\,y^2}{dt}\right) - mgy  
\end{equation}
We can see this is unchanged by moving the $x$ axis by a quantity $\epsilon$, translated by the $x^\prime$ transformation,
\begin{equation}
x^\prime \longrightarrow x + \epsilon \implies \frac{d\,x^\prime}{dt} \longrightarrow \frac{d\,x}{dt}  
\end{equation}
by checking the current, 
\begin{equation}
j= \frac{ \partial \mathcal{L}}{ \partial \frac{d \phi}{dt}} \delta \phi = m \frac{d\,x}{dt}
\end{equation}
is also conserved. We know this to be form of the momentum in the x-direction which we expected to be conserved in this problem. A more laborious exercise could show that conservation of energy comes from the invariance of an action under translations in time. And even things like conservation of charge, which are a little more complicated, come from this symmetry principle. 

\subsubsection{In Minkowski space}

In the normal 3+1 dimensional space the form of Noether's current changes to, 
\begin{equation}
\partial_\mu j^\mu = 0 \implies \frac{ \partial j^0 }{\partial t} + \nabla \cdot \mathbf{j} = 0  
\end{equation}
where we usually call $j^0$ the charge density while $\mathbf{j}$ is named the current density. 

\subsubsection{Classical Electrodynamics}
maybe if there is space  

\subsubsection{Gauge Transformations}

\subsection{Higgs Mechanism}

\subsection{Composition of the Standard Model}

{ \color{red}
The Standard Model is composed by force carriers, the weak gauge bosons W and Z, the photon, the electromagnetic interaction messenger and the strong force mediators, the gluons, as well by matter particles, the quarks and leptons. Being that the Higgs boson is responsible for the mass generation mechanism.

Fermions are organized in three generations. Furthermore, there are 6 different types of quarks, up and down for the first generation, charm and strange for the second as well as top and bottom for the third one. Similarly, there are 6 types of leptons, the charged ones, electron, muon and tau, and the associated neutrinos, respectively represented by $(u,d,c,s,t,b)$ while leptons as $(e,\nu_{e},\mu,\nu_{\mu},\tau,\nu_{\tau})$

So far we have described the physical states that are often denoted as the building blocks of nature. However we have not yet explained how such states have acquired their masses and gauge quantum numbers, such as colour and electric charge. To see this, we start by noting the the SM is a gauge theory based on the group.

\begin{equation}
SU(3)_c \times SU(2)_L \times U(1)_Y \quad  .
\label{SMsymmetry}
\end{equation} 

Fermions are half integer spin particles most of which have electrical charge (except the neutrinos).  While quarks interact via the weak, electromagnetic and strong forces, the charged leptons only feel the electromagnetic and weak forces and the neutrinos are solely weakly interacting.  

A physical fermion is composed of a left-handed and a right-handed part. While the former transform as $SU(2)_L$ doublets and can be written as,

\begin{equation}
L^i= \begin{pmatrix}
\nu_{e_L} \\ e_L 
\end{pmatrix},
\begin{pmatrix}
\nu_{\mu_L} \\ \mu_L 
\end{pmatrix},
\begin{pmatrix}
\nu_{\tau_L} \\ \tau_L 
\end{pmatrix} 
\quad 
\text{and} \quad Q^i= \begin{pmatrix}
u_{L} \\
d_L 
\end{pmatrix},\begin{pmatrix}
c_{L} \\
s_L 
\end{pmatrix}
,\begin{pmatrix}
t_{L} \\
b_L 
\end{pmatrix} \quad ,
\end{equation}

}

\subsection{Quantum fields}

\subsubsection{Spin-0 Fields}


Equation of Motion for Scalar Fields

Lagrangian for Scalar Fields

Solutions to the Klein-Gordon Equation

\subsubsection{Spin-1/2 Fields}

Spinors 

The Action for a Spin 1/2 Field

Parity and Handedness

Weyl Spinors in Any Representation

Solutions to the Dirac Equation

\subsubsection{Gauge Theory}

Conserved currents 

The Dirac Equation with an Electromagnetic Field

Gauging the Symmetry / Charge conjugation

\subsection{Anomaly cancellation}



