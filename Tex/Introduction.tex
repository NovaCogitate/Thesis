%\chapter{Introduction}
%\label{ch:Intro}

\newpage

\section{Introduction}

Although the current theory widely used in the study of particle physics is the Standard Model (SM). A theory that has been shown to be thus far the best descriptor for the experimentally observed spectra of particles and their interactions at all probable scales. The SM boasts being one of the most monumental achievements in physics as it combined three of the four fundamental forces of nature in a very well motivated framework and now that it's last fundamental component, the Higgs Boson as been discovered, we can finally attest that the mechanism of spontaneous symmetry breaking of the $SU(2)_L \times U(1)_Y$ into the electromagnetic symmetry $U(1)_{EM}$ generates all physical masses for the fermions and mediator bosons. 

However, despite all these successes the SM still lacks a strong theoretical explanation for several experimental observations. Firstly, the SM can not account for one of the most important cosmological discoveries of the century observed trough gravitational lensing and many other experiments, the existence of dark matter. This is a fundamental flaw since the SM lacks a possible dark matter candidate, or dark particle. 

Secondly, the SM lacks any justification for the existence of baryon asymmetry in the universe, i.e. why is the universe primarily made of matter rather than anti-matter. Although note that the Electroweak  baryogenesis (EWBG) remains a theoretically possible and experimentally testable {\color{blue} in future?} scenario for explaining the cosmic baryon asymmetry, a scenario viable in the SM framework {\color{blue} is it? what about CP violation?}. As its name suggests, EWBG refers to a mechanism that produces an asymmetry in the density of baryons decaying during the electroweak phase transition. {\color{red} As any of this been tested by GW experiments?}. This puts some requirements on the composition of the universe but would imply that all matter anti-matter asymmetry is created during the time when the Higgs field is settling into it's new vacuum expectation value (VEV). 

Thirdly, the SM has particular oddities in the fermion sector where peculiar and more importantly unjustified mass and mixing hierarchies occur. This is usually refereed to as the \textit{flavour problem} and is considered a drawback of accepting the SM as is. Specifically, we observe the top quark to be five order of magnitudes heavier the up quark, and eleven orders of magnitude than the observed neutrino masses. It is believe by many that eleven orders of magnitude is far too high of a gap to be merely it's nature, so a mechanism that would justify such gap is a desired property of most Beyond the Standard Model frameworks. 

Fourth, note neutrino masses are not included by any mechanism in the 
SM. Although there are precise oscillation measurements that show it's masses in the eV range with precise mixing in between 3 different generations. 

These are just some of the typical justifications given to explore possible BSM scenarios. The holy grail of which would be a model that include all these problems in a properly motivated framework that addresses both cosmological, gravitational, and phenomenological problems. {\color{blue} I should mention cosmos}. 

However the available space for new physics gets reduced by each successful experiment, being one of the reasons the SM remains so prevalent that it has shown puzzling consistence with some measurements that were initial believed to be a possible gateway to new physics. Thus the search continues for hints at possible directions to complete the SM. {\color{blue} I should mention flavour changing }  

Chief among these experiments is the Large Hadron Collider (LHC), whose large amount of collect data over these past years is setting more and more stringent bounds on viable parameter spaces of popular BSM scenarios. And as available space for new physics decreases it becomes more challenging to reveal remaining space without falling within the possibility of fine tuning our model.  
%
{\color{blue} How to properly explain what fine tunning is?}

Conventionally, BSM searches in these multi-dimensional parameter spaces have often been made in large computer-clusters with use of several weeks of computational time trough simple Monte-Carlo methods. Although this is the basis of the work presented here a effort was made to incorporate new machine learning routines trough the initial building of smaller learning sets trough conventional methods. Unfortunately this wasn't accomplished in this work.

During this work we shall do a small expedition into possible BSM scenarios. To begin we discuss possible extensions of scalar sectors that can be embedded into the SM while also examining the consequences of those addition in the other sectors. Note while the minimally structure of the Higgs sector postulated by the SM is not a immediate contradiction of measurements. It is not manifestly required by the data. 

This is done as a exercise to observe if such additions are viable despite the relatively tight bounds on Higgs boson couplings to SM gauge boson and heavy fermions. {\color{blue} how much should I mention of the 3HDM and the BLSM in the introduction?}

We give a higher repute to the Higgs Sector since fermion masses and mixing patterns relate often to the specific structure of the Higgs sector. Also the addition of new scalars offer a large playground for colliders and often the inclusion of new neutrino physics. Models with more than one Higgs doublet also addresses the observed charge parity CP violation with the drawback of potentially having large FLavor Changing Neutral Currents (FCNCs). These FCNCs are undesirable at least in large number given observations and multiple Higgs Doublets could include these diagrams at tree-level, very problematic. A easier model would be a simple addition of a unitary symmetry.

Two particular multi-Higgs extensions will be presented in this work a phenomenological study of a 3 Higgs Doublet model (3HDM) with softly broken $U(1) \times Z_2$ symmetry and a simple Unitarity ($U(1)$) extension of the SM based on the apparent Baryon minus Lepton symmetry (BLSM) model.

