%\chapter{Introduction}
%\label{ch:Intro}

\newpage

\chapter{Introduction}
%\section{Introduction}

Modern study of particle physics is often taught trough the Standard Model (SM) of particle physics. The SM has thus far the best descriptor for the experimentally observed spectra of particles and their interactions at all current probable scales. And In 2012 a resonance was discovered in the LHC that seems to confirm the existence of it's last predicted particle, the Higgs boson, finally completing the Model and proving the existence of the Higgs mechanism. 

%Without a doubt the conventional theory which is to be used in the conventional study of particle physics is the Standard Model (SM) of particle physics. Given the fact that the SM has thus far the best descriptor for the experimentally observed spectra of particles and their interactions at all current probable scales. 

The history of the SM is a interesting one, and with it, scientists combined three of the four fundamental forces of nature in a very well motivated framework, making it one of the most monumental achievements in physics. 

%This discovery validated the mass generation mechanism of all subatomic particles as stemming spontaneous symmetry breaking (SBB). The SM is based on the gauge group $\mathrm{SU(3)_C \times SU(2)_L \times U(1)_Y}$  where the broken symmetry are that of the the $\mathrm{SU(2)_L \times U(1)_Y}$ into the electromagnetic symmetry $\mathrm{U(1)_{Q}}$. {\color{blue} shorten?}

% { \color{blue} Is it safe to say that the SM is more of a empirical model given how we set it's couplings and the such and mention it doesn't account for GUT for example?}

However, despite all these successes the SM still lacks a strong theoretical explanation for several experimental observations. To name a few, firstly, the SM can not account for one of the most important cosmological discoveries of the century {\color{gray}, observed trough gravitational lensing,}  the existence of dark matter. This is a fundamental flaw since the SM lacks a possible dark matter candidate, or dark particle. 

Secondly, the SM lacks any justification for the existence of baryon asymmetry in the universe, i.e. why is the universe primarily made of matter rather than anti-matter. Although note that the Electroweak baryogenesis (EWBG) remains a theoretically possible scenario for explaining the cosmic baryon asymmetry, a scenario viable in the SM framework. {\color{blue} Is this stil true?}

{\color{gray} As its name suggests, EWBG refers to a mechanism that produces an asymmetry in the density of baryons decaying during the electroweak phase transition. This puts some requirements on the composition of the universe but would imply that all matter anti-matter asymmetry is created during the time when the Higgs field is settling into it's new vacuum expectation value (VEV). } 
{\color{blue} Should I remove this paragraph?}

Thirdly, the SM has peculiar oddities in the fermion sector in the form of unjustified mass and mixing hierarchies. This is usually refereed to as the \textit{flavour problem} and is considered a big drawback of the SM. { \color{gray} Specifically, we observe the top quark to be five order of magnitudes heavier the up quark, and eleven orders of magnitude than the observed neutrino masses.} These high differences are thought to be too large to be simply "nature", so a physical property that would justify such gap is a desired property of most Beyond the Standard Model (BSM) frameworks. 

Fourth, note neutrino masses are not included in the 
SM. Although there are precise oscillation measurements that measure masses in the eV range with precise mixing in between 3 different generations of neutrinos. 

These are just some of the typical justifications given to explore possible BSM scenarios. The holy grail of which would be a model that include all these problems in a properly motivated framework that addresses these and many more cosmological, gravitational, and phenomenological problems.  

However, as of late the research in possible BSM scenarios as become progressively harder to perform given that the available space for new physics gets reduced by each successful experiment. Chief among these experiments is the Large Hadron Collider (LHC), whose large amount of collect data over these past years is setting more and more stringent bounds on viable parameter spaces of popular BSM scenarios. And as available space for new physics decreases it becomes more challenging to reveal remaining space without falling within the possibility of fine tuning our model.  

{\color{blue} How to properly explain what fine tunning is? Should I?}

Note, that the SM has shown increasingly puzzling consistence with most constraints that were initial believed to be a possible gateway to new physics (NP) or that would diverge from it. Thus, the search continues for hints at possible directions to complete the SM. {{\color{blue} I should mention flavour changing and how the SM needs a very strange matrix for that to happen}

Conventionally, BSM searches in these multi-dimensional parameter spaces have often been made in large computer-clusters with use of several weeks of computational time trough simple Monte-Carlo methods. Although this is the basis of the work presented here a effort was made to incorporate new machine learning routines trough the initial building of smaller learning sets trough conventional methods. Unfortunately this wasn't accomplished in this work due to the expectational setbacks felt this year.

During this work we shall do a small expedition into possible BSM scenarios. However we will start by laying down the fundamental basis for this BSM discussion by presenting a short overview of the SM, then we discuss possible extensions and alterations of the SM. First by presenting the B-L-SM model, a simple unitary extension based on a apparently accidental symmetry of the SM. And then by moving on to a more complex model with additional Higgs doublets fields as a attempt to present a framework that addresses the \textit{flavour problem}. Note while the minimally structure of the Higgs sector postulated by the SM is not a immediate contradiction of measurements. It is not manifestly required by the data. And in fact a extended scalar sector is often desired to deal with many of it's shortcomings despite the relatively tight bounds on Higgs boson couplings to SM gauge boson and heavy fermions. We give a higher repute to the Higgs Sector since fermion masses and mixing patterns relate often to the specific structure of the Higgs sector. Also, the addition of new scalars offer a large playground for collider experimentation and often offer the inclusion of new neutrino physics. 

The models with more than one Higgs doublet also addresses the observed charge parity CP violation. With the drawback of potentially having large Flavour Changing Neutral Currents (FCNCs). These FCNCs are undesirable at least in large number given observations, although multiple Higgs Doublets could include these diagrams at tree-level, making them very problematic. { \color{blue} The mechanisms that suppress this should be presented. } 

In short two particular multi-Higgs models will be presented in this work a phenomenological study of a 3 Higgs Doublet model (3HDM) with softly broken $\U \times \mathrm{Z_2}$ symmetry and a simple Unitarity, $\mathrm{U(1)}$, extension of the SM based on the apparent Baryon minus Lepton symmetry (B-L-SM). We'll investigate what can be learned from these models and what other physical experiments constrict them. 

