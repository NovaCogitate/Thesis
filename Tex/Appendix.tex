\chapter{Appendix}

\section{Appendix}

\subsection{Gamma Matrices}
The $\gamma$ matrices are defined as, 
\begin{equation}
\{  \gamma^\mu , \gamma^\nu \} = 2 g^{\mu \nu} I 
\end{equation}
where, 
\begin{equation}
g^{ \mu \nu } = 
\begin{pmatrix}
1 & 0 & 0 & 0 \\
0 & -1 & 0 & 0  \\
0 & 0 & -1 & 0 \\
0 & 0 & 0 & -1 \\
\end{pmatrix}
\end{equation}
and if $\gamma_\mu = (\gamma^0, \gamma)$  then it is usual to require for the hermitian conjugate matrices,
\begin{equation}
\gamma^{0 \dagger} = \gamma^0 \quad \textrm{and} \quad \gamma^\dagger = - \gamma 
\end{equation}


\subsection{Lagrangian Dynamics}

In Lagrangian dynamics we define the action $S$ has, 
\begin{equation}
\mathcal{S} = \int L \, dt = \int \mathcal{L}(\phi,\partial \phi ) \, d^4x  
\end{equation}
where $L$ is the Lagrangian, and the $\mathcal{L}$ is designated as the \textit{Lagrangian density}, note these terms are
usually used interchangeable. Here $\mathcal{L}$ is a function of the field $\phi$ and it's spatial derivatives. 

The action $S$ is constrained by the principle of least action, this requires the "path" taken by a field between an initial and final set
of coordinates to leave the action invariant, this can be expressed by,
\begin{equation}
\partial \mathcal{S} = 0 
\end{equation}
from here one can deduce the \textit{Euler-Lagrange} equations,
\begin{equation}
\partial_\mu \left( \frac{\partial \mathcal{L}(\phi,\partial \phi ) }{ \partial (\partial_\mu) } \right) - \frac{ \partial \mathcal{L}(\phi,\partial \phi ) }{ \partial \phi }  = 0 
\end{equation} 