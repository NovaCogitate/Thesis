%\chapter{Introduction}
%\label{ch:Intro}

\newpage

\chapter{Introduction}
\label{chapter:Introduction}
%\section{Introduction}

%\section{The Sturdy SM with some holes}

\Joaorep{The modern study of particle physics}{Our current understanding of all subatomic phenomena} must be \Joaorep{taught}{understood} trough the Standard Model (SM) of particle physics. 
%
The SM has thus far \Joaoadd{been} the best descriptor for the experimentally observed spectra of particles and their interactions at \Joaorep{all current probable scales}{the electroweak (EW) scale}. 
%
And \Joaorep{In}{in} 2012\Joaoadd{,} a resonance was discovered \Joaorep{in}{at} the \Joaorep{LHC}{Large Hadron Collider (LHC)} that seems to confirm the existence of \Joaorep{it's}{its} last predicted particle, the Higgs boson, finally completing the model and proving the existence of the Higgs mechanism \cite{Aad_2012,chatrchyan2012observation,
collaborations2015combined,collaborations2016measurements} \Joao{Não necessitas de tantas referências para o Higgs, uma ou duas basta}. 

The development of the SM was a arduous task, it led scientists  \Joaoadd{to} successfully combine three of the four fundamental forces of nature in a \Joaoout{very} well motivated framework, making it one of the most monumental achievements in theoretical physics.
%
However, despite \Joaorep{it's}{its} successes the SM still lacks a strong explanation for several experimental observations. \Joaoout{They have become more numerous by the decade, and to provide a "short" overview of some of them.}

\Joaorep{We firstly}{First, we} have the fact \Joaoadd{that} the SM can not account for one of the most important cosmological discoveries of the century, the existence of dark matter \Joao{Ref.}. This is a fundamental flaw since the SM lacks a possible dark matter candidate, or dark particle \Joao{Ref.}. 
%
Secondly, the SM lacks any justification for the existence of baryon asymmetry in the universe \Joao{Ref.}, i.e. why is the universe primarily made of matter rather than anti-matter. 
%
Although, \Joaoout{note that} the Electroweak baryogenesis (EWBG) remains a theoretically possible scenario for explaining the cosmic baryon asymmetry \Joao{Ref.}, a scenario viable in the SM framework.
%
Thirdly, the SM suffers from peculiar oddities in the fermion sector in the form of unjustified mass and mixing hierarchies. This is usually refereed to as the \Joaorep{\textit{flavour problem}}{flavour problem} and is considered a sizeable drawback of the SM .  
%
As \Joaorep{a}{an} example, we observe the top quark to be five order of magnitudes heavier \Joaoadd{than} the up quark , and eleven orders of magnitude than the observed neutrino masses \Joao{secalhar põe aqui os valores das massas, fica mais elucidativo, ou pelo menos, a ordem. Para o top $\mathcal{O}(100)$ GeV, o up $\mathcal{O}(1)$ MeV e os neutrinos $\mathcal{O}(1)$ eV}. These high differences are thought to be too large to be natural, so a physical property that would justify such gap is a desired property of most Beyond the Standard Model (BSM) frameworks. 
%
Fourth, \Joaoout{note} neutrino masses are not included in the SM. Although there are precise oscillation measurements \Joao{Ref.} that \Joaorep{measure}{estimate} masses in the eV range with precise mixing \Joaoout{in} between 3 different generations of neutrinos \Joao{Uma nota aparte, as oscilações dos neutrinos dependem da diferença de massas entre gerações $\Delta m$ e não da massa absoluta de cada um, ou seja, um deles pode não ter massa, mas um outro tem de ter}. 
%
There are still many other subtitle flaws, like the lack of a strong phase transition, \Joaoadd{the hiearchy problem, the $R_\kappa$ and g-2 anomalies,} etc.  \Joao{Mencionar pelo menos a anomalia g-2, pois analisas mais à frente no B-L-SM} 

These are just some of the typical justifications given to explore possible BSM scenarios. The holy grail of which would be a model that \Joaorep{include}{solves} all \Joaoadd{of} these problems in a properly motivated framework that addresses these and many more cosmological, gravitational \Joao{Gravitational? O que queres disser aqui? Não falas de gravidade quântica na tese.} and phenomenological problems.  
% 
For now\Joaoadd{,} such a model remains out of reach, so the narrowing down of theories through phenomenological studies is a very worthwhile endeavour. We try to present one of these studies in this work. % to the steady advancement of a more complete theory. 
%
Paradoxically \Joaoout{fortunately}, as of late\Joaoadd{,} these studies have become progressively harder to perform given that the available space for new physics \Joaoadd{(NP)} gets reduced by each successful particle experiment. 
%
Chief among \Joaorep{these experiments}{them} \Joaorep{is the}{are the CMS and ATLAS experiments at the} \Joaoout{Large Hadron Collider} LHC, whose large amount of collect data over past years is setting \Joaorep{more and more stringent}{ever stringent} bounds on viable parameter spaces of popular BSM scenarios. 
%
And as \Joaoadd{the} available space for \Joaorep{new physics}{NP} decreases\Joaoadd{,} it becomes more challenging to reveal \Joaoadd{the} remaining space without falling within the possibility of fine tuning our model.  

%{\color{blue} How to properly explain what fine tunning is? Should I?}

Note, that the SM has \Joaorep{shown increasingly}{increasingly shown} \Joaorep{consistence}{consistency} with most constraints that were initial believed to be a possible gateway to \Joaoout{new physics} NP or that would diverge from \Joaorep{it's}{its} predictions. Thus, the search continues for hints at possible directions to complete the SM. % One of these is brought to use trough flavour physics, as we'll soon examine further bellow. 

Conventionally, phenomenological simulations of BSM searches in these multi-dimensional parameter spaces have been made in large computer-clusters, \Joaorep{with use of}{requiring} several weeks of computational time trough \Joaorep{simple}{the use of} Monte-Carlo methods. 
%
Although this is the basis of the work presented here\Joaoadd{,} a effort was made to incorporate new machine learning routines \Joaorep{trough}{via} the initial building of smaller learning sets \Joaorep{trough}{by} conventional methods. 
%
Unfortunately this \Joaorep{wasn't}{was not} accomplished in this work due to the \Joaorep{expectational}{exceptional} setbacks. A feature of this year, that \Joaorep{affected partially}{partially affected} the quality of the work. 

\Joao{Como não fazes machine learning, acho que não te devias referir a isto.}

%%%% so far so good

During this \Joaorep{work}{thesis} we \Joaoout{shall} \Joaorep{do}{embark in} a small expedition into two possible BSM scenarios.
%
To achieve this, we will start by laying down the fundamental basis for this BSM discussion by presenting a short overview of the SM \Joaorep{, then we discuss possible extensions}{followed by a discussion into potential extensions of this framework.}\Joaoout{ to the SM}. \Joaorep{Namely, first by presenting}{First, we introduce} the B-L-SM model, a simple unitary extension based on a apparently accidental symmetry of the SM.  \Joao{Que simetria? Acho que ficava bem mencionar aqui}.
%
\Joaorep{And then by moving on}{Then, we move on} to a more complex model with additional Higgs doublets fields as \Joaorep{a}{an} attempt to present a framework that addresses the \Joaorep{\textit{flavour problem}}{flavour problem}. 
%
We will see how these \Joaoout{multiple doublet} \Joaorep{Models}{models} can address problems that \Joaoadd{a} simple unitarity extension \Joaorep{can't}{can not} and vice-versa. For example\Joaoadd{,} multiple Higgs \Joaorep{Doubles}{doublets} can easily offer \Joaorep{a}{an} explanation for the observed excess of charge parity or $\mathcal{CP}$ violation \Joaoadd{b}ut suffer from the possible inclusion of tree-level Flavour Changing Neutral Currents (FCNCs). These FCNCs are undesirable\Joaoadd{,} at least in large number\Joaoadd{,} given \Joaoadd{current} observations \Joao{Refs.}, so mechanisms have to be put in place to prevent them, while in the case of the simple unitary extensions such problems do not arise. 

%We will see how these models with more than one Higgs doublet can address yet another, thus far, unmentioned problem in the SM, the observed excess of charge parity or $\mathcal{CP}$ violation.  
%
%While suffering the possible the drawbacks of potentially having large Flavour Changing Neutral Currents (FCNCs). These FCNCs are undesirable at least in large number given observations, although multiple Higgs Doublets could include these diagrams at tree-level, making them very problematic. We present a specific version of a Multiple Higgs Doublet, specifically a 3HDM model with a symmetry mechanism that will suppress these FCNCs. 

I also want to stress that, while the \Joaorep{minimally}{minimal} structure of the Higgs sector postulated by the SM is not \Joaorep{a}{an} immediate contradiction \Joaorep{of}{to experimental} measurements. It is not manifestly required by the data. \Joaorep{And in fact}{In fact,} \Joaorep{a}{an} extended scalar sector is often \Joaorep{desired}{desirable,} despite \Joaoout{the  relatively} tight bounds on \Joaoadd{the} Higgs boson couplings to SM gauge boson\Joaoadd{s} and heavy fermions. 
%
These additions are motivated \Joaoout{also in part} by \Joaoadd{the fact that} in the  SM,  the  single Higgs  doublet is  a bit "overstretched".  It  takes  care\Joaoadd{,} simultaneously\Joaoadd{,}  of the \Joaorep{masses of the gauge bosons}{gauge bosons masses} \Joaorep{and of}{,} the up and down-type \Joaorep{fermions}{quarks} \Joaorep{and}{as well as the} leptons. N-Higgs-doublet models \Joaorep{and}{have multiple} scalar or complex fields \Joaoadd{that} relax  this  requirement.   
%
In particular\Joaoadd{,} \Joaoout{the} multiple Higgs doublet models are \Joaorep{based "natural"}{are engineered based on naturalness arguments, that is,}\Joaoout{,  suggestion,  that} the  notion  of  generations  can  be  brought  to  the  Higgs  sector.
\Joao{Esta última frase está um bocado confusa. Acho que a devias reescrever.}
\\ \\
\Joao{Algumas notas gerais, eu notei que estas a confundir o uso de ``a'' e ``an''. Usas ``a'' quando a palavra seguinte começa com uma consoante, usas ``an'' quando começa com uma vogal. Também notei que trocas ``it's'' com ``its''. Lembra-te que ``it's'' é uma contração de ``it is''. Em relação a contrações, nunca as uses, pois eu notei que usaste em alguns casos.} 
\\ \\ 
\Joao{O principal problema são as referências, não tens quase nenhumas. Eu no texto coloquei locais onde deves pôr referências. Secalhar em outros locais possa ser necessário. Lembra-te, se não é novo ou feito no contexto desta tese, então deve ter referência. Mesmo assim, estás com quase 80 referências. Tenho que ver se todas são necessárias. Notei que tinhas o PDG repetido, já corrigi. É possível que aja outras. O facto de usares vários ficheiros .tex e .bib é confuso. Também notei que as referências não começam em [1]. A referência [1] só aparece mais tarde. Existe um bug qualquer.}
 \\ \\ 
 \Joao{Também notei inconsistência na definição de acronónimos. As vezes não defines ou defines demasiado tarde. Tenta garantir sempre que, no primeiro momento que aparece no texto, o acronónimo é definido.} \\ \\
\Joao{Em geral, não está assim tão mau. Eu tentei não alterar o sentido que queres passar mas o Morais e o Felipe são capazes de fazer algumas mudanças.}\\ \\
\Joao{Notei que a notação não é consistente. Tenta ter cuidado com isto.}
\\ \\
\Joao{Outros detalhes eu vou mencionado pelo texto.}
%Both these extensions have the bonus of leading to remarkably rich  phenomenology (for a detailed review, see e.g. Refs. ( \cite{branco1999cp,Branco_2012,Ivanov_2017} ). And in general BSM scenarios offer features  as  several  Higgs  bosons,  charged  and neutral, modification of the SM-like Higgs couplings, FCNC at tree level, additional forms of CP-violation from the scalar sector, and opportunities for cosmology such as scalar DM candidates and modification of the phase transitions in early Universe. {\color{blue} Repeated, Fix later.} Also, many BSM models including supersymmetry (SUSY), gauge unification models, and even string theory constructions naturally lead to several Higgs doublets at the electroweak scale.

%We give a higher repute to the Higgs Sector since fermion masses and mixing patterns relate often to the specific structure of the Higgs sector. Also, the addition of new scalars offer a large playground for collider experimentation and often offer the inclusion of new neutrino physics. 

%In short two particular multi-Higgs models will be presented in this work a phenomenological study of a 3 Higgs Doublet model (3HDM) with softly broken $\U \times \mathrm{Z_2}$ symmetry and a simple Unitarity, $\mathrm{U(1)}$, extension of the SM based on the apparent Baryon minus Lepton symmetry (B-L-SM). We'll investigate what can be learned from these models and what other physical experiments constrict them. 

%The SM extensions featuring non-minimal Higgs sectors with extra Higgs doublets in analogy to fermion generations in the SM provide a fruitful playground for constructing successful BSM scenarios (for a detailed review, see e.g. Refs. \cite{branco1999cp,Branco_2012,Ivanov_2017} ).

%There is also no constraints stemming from the $\rho$ parameter here.  Since all doublets couple to the gauge-bosons in the same way, the W and Z masses are determined by the single value, the sum of real VEVs. Assuming this value is 246 GeV it would retain the condition $\rho = 1 $ at tree-level. 

%Multi-doublet models offer novel opportunities for CP-violation.  Within the SM, it is put by hand coming entirely from the Yukawa matrices which must be complex.  In multi-doublet models, a relative phase between vevs can arise just as a result of the minimization of the potential. Leading to a more natural and spontaneous CP-violation. 

%A real or even complex singlet extension is a also simple pathway to extending the SM. In a generic model with a SM Higgs Doublet the addition of a generic gauge singlet scalar, S, could prove a link between the SM fields and a unknown hidden sector. 
%
%In spite of our ignorance of this hidden sector, we can simply assume a generic renormalizable self-interaction for the scalar S and investigate the joint $(\phi,S)$ potential. This would lead to a generic mixing, $\alpha$ between scalars. 

%In the case the additional scalar field is complex this brings a additional degree of freedom and has the possibility of 3 neutral scalars mixing, depending on the shape of the VEV. Producing  a slightly  richer  collider  phenomenology  and  complicating its analysis.  

%Are heavily constricted from experimental measurements we know that in this framework fermions and gauge bosons should primary couple to $h_\phi$. This type of mixing suppression, ($\alpha < 10^3$), but even so the heavy Higgs can in most cases decay into a pair of light ones if this channel is open kinematically, providing a avenue for detection. 

%On the other hand, one or both new scalars can be symmetry protected against decay, yielding simple models of one or two-component dark matter or models with one DM candidate and a strong electroweak phase transition.

%Before moving on, let us make a remark on the (absence of) CP-violation in the singlet extension of SM. Although the potential contains many complex coefficients, it does not produce CP-violating effects in the scalar sector, see \cite{branco1999cp}.




