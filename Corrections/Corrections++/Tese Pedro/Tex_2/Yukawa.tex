\renewcommand{\cleardoublepage}{}
\renewcommand{\clearpage}{}

\section{Fermion Masses in the SM and Quark mixing}

As referenced \Joaoadd{previously}, given the charges of the fermion and lepton fields\Joaoadd{,} we cannot construct a gauge invariant theory with explicit mass terms for fermions. 
%
The mass of these particles \Joaorep{are generated through couplings to the Higgs, by the Higgs mechanism}{are generated through the Higgs mechanism, via Yukawa terms between the fermions and the scalar field. These interactions can be seen in Eq.\,\eqref{eq:YukawaSM}}. \Joaoout{We can write these interactions as,}
%
%\begin{equation} 
%\label{eq:Yukawa2}
%\mathcal{L}_{Yuk} = Y^u_{ij}\bar{Q}_{i} u_{R_j}  \tilde{H} + Y^d_{ij} \bar{Q}_{i}  d_{R_j} H  + Y^e_{ij} \overline{L_{L_i}}  e_{R_i} H + h.c.
%\end{equation} 
%
\Joaoout{where $Y^{e,u,d}$ stand for the Yukawa matrices, these are generic $3\times3$ complex non-dimensional coupling matrices, $H$ is the Higgs field with $\tilde{H}$ retaining it's previous definition, $i,j$ are the standard generation indices, $Q_{L_i}$,are the left handed quark doublets, while $d_R$ and $u_R$ are the corresponding right-handed down and up quark singlets respectively in the weak eigenstate basis.}
%
\Joaorep{Has}{As} the Higgs field settles into the electroweak VEV \Joaoadd{(see Eq.\,\eqref{eq:vev_expansion})} \Joaoout{yields} mass terms for the quarks and leptons \Joaoadd{are generated}. 
%
The Higgs mechanism generates the mass for all the fermionic and leptonic particles except for neutrinos\Joaoadd{. T}his is due to the SM not containing right handed neutrinos, i.e we \Joaorep{can't}{can not} build terms that would lead to neutrino masses.
% 
The addition of right handed neutrino fields is very commonly made in BSM scenarios. 

To reach the physical states \Joaoout{starting} from the weak eigenbasis \Joaorep{you}{we} must diagonali\Joaoadd{s}e the Yukawa matrices. This is done through \Joaoout{a} bi-unitary transformation\Joaoadd{s}. 
% 
We can write these transformation under the form,
%
\begin{equation}
\label{YukawaMasses} 
M^{u,d,e}_{\text{diag.}}= U^{u,d,e}_L Y^{u,d,e} U^{u,d,e}_R 
\end{equation} 
%
\Joaorep{where $v$ stands for the electroweak VEV. And}{where} $U^{u,d,e}_L$ and $U^{u,d,e}_R$ are \Joaoout{the required} \Joaorep{6}{4} unitary matrices. \Joaoadd{For simplicity, we shall assume that in the leptonic sector the Yukawa matrices are flavour diagonal, hence, only the two unitary matrices for the quarks will be of relevance to our discussing bellow.}
%
\Joaoout{
It is, in fact, these matrices that will get us from the flavour eigenbase to the mass eigenbase. }
%
%{\color{gray} The charged lepton Yukawa matrix can always be made real and positive through a bi-unitarity transformation.  
%
%Meaning the Yukawa matrix for the leptons contains only 3 real physical parameters that correspond to the Lepton masses. } 
%
Naturally\Joaoadd{,} we can invert Eq.\,\eqref{YukawaMasses}, returning \Joaoout{equations for the Yukawa matrices as,} 
\begin{equation}\label{eq:YukawaBiUni}
\begin{aligned}
Y^u_{ij} = &  (U_L^u M^u_{\text{diag.}} U_R^u)_{ij}, \\
Y^d_{ij} = &  (U_L^d M^d_{\text{diag.}} U_R^d)_{ij}.
\end{aligned}
\end{equation}
%
\Joaoadd{Considering now the Higgs mechanism,} we can see this change creates mass terms for physical quark fields by replacing the result of Eq.\,\eqref{eq:YukawaBiUni} in the Yukawa portion of the Lagrangian Eq.\,\eqref{eq:YukawaSM}
%
\begin{gather}
\mathcal{L}_{Yuk} \supset 
- Y^d_{ij} \begin{pmatrix} \overline{u}_{L\,i} & \overline{d}_{L\,i}  \end{pmatrix}  d_{R\,j} \Joaoadd{\tilde{H}} 
%
- Y^u_{ij} \begin{pmatrix} \overline{u}_{L\,i} & \overline{d}_{L\,i}  \end{pmatrix} \, u_{R\,j} H + \Joaoadd{\mathrm{H.c.}} \nonumber  \\ 
% % % 
 \Downarrow \nonumber \\
-(U_L^d m^d_{\text{diag.}} U_R^d)_{ij} d_{L\,i} \, d_{R\,j}  - (U_L^u m^u_{\text{diag.}} U_R^u)_{ij} u_{L\,i} \, u_{R\,j} + \big(\text{Interactions \Joaoadd{with $h$}}\big) + \Joaoadd{\mathrm{H.c.}} \\ 
 \Downarrow  \nonumber \\ 
-m^d_{\text{diag.}_j} d_{L\,i}^\prime \, d_{R\,j}^\prime  - m^u_{\text{diag.}_j} u_{L\,i}^\prime \, u_{R\,j}^\prime + \big(\text{Interactions \Joaoadd{with $h$}}\big) + \Joaoadd{\mathrm{H.c.}}  \nonumber  
\end{gather}
%
\Joao{Hermitian é em homenagem ao matemático Charles Hermite, portanto tem que estar em maiúsculo, H.c. e não h.c.} where the primed fields are the quark fields in the mass basis, defined as 
\begin{equation}
\begin{split}
d^\prime_{L,R} = U^d_{L,R} d_{L,R}, \\
u^\prime_{L,R} = U^u_{L,R} u_{L,R}.
\end{split}  
\end{equation}
% 
Note that the increasing masses seen in each generation depend directly on the \Joaoout{term} hierarchy of the Yukawa terms. This means that the mass of all particles directly relate\Joaoadd{s} to how strongly they each interact with the Higgs boson.
%
If you then take into account the real masses e.g. for the leptons, the tau mass is in the GeV range while the electron's is in the 0.1 MeV range. \Joaorep{These translate}{This translates} to very different couplings for each flavour. 
%
This hierarchy is unjustified in the SM. 

As a result of this redefinition, we can now look at the gauge interactions to see that \Joaoout{a} charge\Joaoadd{d} current\Joaoadd{s} appear\Joaoout{s} where $W^\pm$ couples \Joaoadd{to} the physical $u^\prime_{L_j}$ and $d^\prime_{L_j}$ \Joaoadd{quarks}. 
%
The coupling of the \Joaoout{of} fermions to \Joaorep{their respective}{the} gauge fields changes by virtue of the fact \Joaoadd{that} only left handed \Joaoadd{q}uarks are $\mathrm{SU(2)_L}$ doublets\Joaoadd{. I}f we expand the up and down quark fields on the kinetic portion of the Lagrangian
%
\begin{equation}\label{LagFermFCCCs}
\begin{aligned}
\mathcal{L}_{ferm} \supset & 
\frac{1}{2} \Joaoadd{\bar{u}}^\prime_L \gamma^\mu \left( g^\prime \Joaoadd{Y} B_\mu + g \Joaoadd{Z}_\mu  \right) \left(U^u_L U^{u \dagger}_L \right) u^\prime_L - \frac{1}{\sqrt{2}} g \Joaoadd{\bar{u}}^\prime_L \gamma^\mu \left( U^u_L U^{d \dagger}_L \right) d^\prime_L W^+_\mu \\   
- 
& \frac{1}{\sqrt{2}} g \Joaoadd{\bar{d}}^\prime_L \gamma^\mu \left( U^u_L U^{d \dagger}_L \right) u^\prime_L W^-_\mu 
+ 
\frac{1}{2} \Joaoadd{\bar{d}}^\prime_L \gamma^\mu \left( g^\prime \Joaoadd{Y} B_\mu - g \Joaoadd{Z}_\mu \right) \left( U^d_L U^{d \dagger}_L \right) d^\prime_L.  
\end{aligned}
\end{equation}
%
\Joaorep{We can trough the use of the properties of unitary matrices}{By employing properties of unitary matrices}, namely, $ \mathrm{U}^{u,d}_{L,R} \mathrm{U}^{u,d \dagger}_{L,R} = 1$, \Joaoadd{we} note that the interactions with the neutral bosons remain the same in the mass basis.
%
However the charged currents are affected by this change.
%
Therefor, we define the Cabibbo-Kobayashi-Maskawa (CKM) matrix, as $V_{CKM} = U^u_L U^{u ^\dagger }_R $ and write the sensitive terms,
%
\begin{equation}
\mathcal{L}_{kin} \supset \frac{1}{\sqrt{2}} g \overline{u}^\prime_L \gamma^\mu V_{CKM} d_L^\prime W^+_\mu + h.c. 
\end{equation}
%
The CKM matrix, is a $3 \times 3$ unitary matrix. It is a parametrization of the three mixing angles and CP-violating KM phase. There are many possible conventions to represent the CKM matrix.\Joao{Ref. Secalhar o PDG, que eles têm uma secção dedicada a isto das diferentes convenções e fases.}
%
The mixing angles refer to those between the up and down quark families. We can see their hierarchy in Fig. \ref{fig:QuarkCKM}.

It is through this complex phase in the CKM matrix that the SM can account for the phenomena of $\mathcal{CP}$ violation. First observed in the famous $K^0$ decay into $\mu^+$ $\mu^-$ ($CP=+1$ and $CP=-1$ respectively\Joaoadd{, see Fig.~\ref{fig:Kaon}}) that won the 1980 Nobel Prize \Joao{Ref}. The discovery opened the door to questions still at the core of particle physics and of cosmology today \Joao{Esta frase não soa bem, acho melhor que a reescrevas}. Not just the lack of an exact CP-symmetry, but also the fact that it is so close to a symmetry. {\color{blue} citation}

\begin{figure}[H]
	\centering
	\includegraphics[width=0.5\textwidth]{Glashow-Illiopoulos-Maiani_mechanism_fig1.png}
	\caption{Box diagram describing $K_L^0\rightarrow\mu^-\mu^+$, through an intermediate $u$ quark. \Joao{A imagem tem pouca qualidade. Consegues fazer isto no latex com a package feynmp. No tutorial que te enviei tem lá o código}}
	\label{fig:Kaon}
\end{figure}

\Joaoout{We avoided discussing leptons since in the SM their mass eigenstates can be easily shown to have no real consequence besides a change of basis.} 
%
\Joaorep{You}{We} might also note a very interesting feature of the Standard Model, by consequence of the $\mathrm{SU(2)_L \times U(1)_Y }$ symmetry\Joaoadd{. T}here are no interactions of the right handed unitary matrices and \Joaorep{there for}{thus} no mixing, coupling, or charged currents of right handed quarks, making them theoretically invisible to measurements. \Joao{Já notei que às vezes usas ``you''. Não te dirigas para o arguente ou para que está a ler. Tenta usas sempre expressões com ``We''}
%
\begin{figure}[H]
	\centering
	\includegraphics[width=0.9\textwidth]{TestYukawaCouplings.pdf}
	\caption{\Joaoout{The} Feynman diagrams for flavour conserving couplings of quarks to photon, $Z$ boson, gluon and the Higgs (the first three diagrams), and the flavour changing coupling to the $W$ (the last diagram). The $3\times3$ matrices are visual representations of couplings in the generation space, with couplings to $\gamma$, $Z$, $g$ \Joaoadd{being} flavour universal, \Joaorep{the}{while} couplings to the Higgs flavour \Joaoadd{are} diagonal but not universal\Joaorep{, and the couplings to}{. The couplings involving} $W$ \Joaoadd{are} flavour changing and hierarchical.}
	\label{fig:QuarkCKM}
\end{figure}
%

% Possible power gap
%
% Here we introduce the CKM matrix, a $3 \times 3$ unitary matrix. It is a paramatrization of the three mixing angles and CP-violating KM phase. There are many possible conventions to represent the CKM matrix. 

The CKM matrix elements are fundamental parameters of the particle physics, so their precise determination is important, and reproducing the quark mixing parameters is fundamental for BSM searches that include changes to how the quarks interact with possible new Higgs bosons. 
