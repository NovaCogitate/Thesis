\chapter{Conclusions}
\label{ch:Conclusions}

%\section{The B-L-SM Conclusions}
%\label{sec:Conclusions BLSM}

To summarise, in this thesis we have performed a detailed phenomenological analysis of the minimal $\U{B-L}$ extension of the Standard Model known as the B-L-SM and the a BGL-like 3HDM with a \Joaoadd{$\mathrm{U(1)} \times \mathbb{Z}_2$} symmetry. 
%
This phenomenological analysis was produced by a set of tools developed as to be easily adaptable to fit new models. 

In the B-L-SM (chapter \ref{Chap:B-L-SM_Model}), we have confronted the model with the most recent experimental bounds from the direct $Z^\prime$ boson and next-to-lightest Higgs state searches at the LHC.
%
Simultaneously, we have analysed the prospects of the B-L-SM for a consistent explanation of the observed anomaly in the muon anomalous magnetic moment $(g-2)_{\mu}$. 
%
Done by exploring the B-L-SM potential for the observed $(g-2)_{\mu}$ anomaly in the regions of the model parameter space that are consistent with direct searches and EW precision observables.

As one of the main results of our analysis, we have found phenomenologically consistent parameter space regions that simultaneously fit the exclusion limits from direct $Z^\prime$ searches and can explain the muon $(g-2)_{\mu}$ anomaly. 
%
We have distinguished four benchmark points for future phenomenological exploration at experiments, the first one with the lightest allowed $Z^\prime$ ($m_{Z^\prime}>3.1$ TeV), the second with the lightest additional scalar boson ($m_{h_2}>400$ GeV), and the other two points that reproduce the muon $(g-2)_{\mu}$ anomaly within $1\sigma$ uncertainty range. 
%
Besides, we have studied the correlations of the $Z^\prime$ production cross section times the branching ratio into a pair of light leptons versus the physical parameters of the model.
%
In particular, we have found that the muon $(g-2)_{\mu}$ observable dominated by $Z^\prime$ loop contributions lies within the phenomenologically viable parameter space domain. 
%
For completeness, we have also estimated the dominant contribution from the Barr-Zee type two-loop corrections and found a relatively small effect.

%%%%%% 

As for the 3HDM portion of our work in this these seen in  Chapter\,\ref{ch:3HDM}. 
%
We verified the phenomenological consistency of our model, we identified a region where both flavour and scalar sector physics are within experimental bounds including, like in the B-L-SM, EW precision observables and direct detection bounds.  
%
%We noted that there is a large region of the paramater space that is consistent with flavour QFV observables and where they are not respected. 

We determined the most sensitive flavour violation channels, and concluded that the stabilizing flavour symmetry is a mechanism that allows the model to be consist with current flavour observations. 
%
We also recognize that further deviations in these flavour observables might be a pathway as to discover Higgs boson mediating these phenomena. 
%
Being that the most appropriate channels to peer into this could be B meson oscillations and B meson decays.

However a key take away from our results we show that the most stringent constraint on the model is not the flavour observables but the unitarity and Higgs physics limits. 
%
A significant conclusion seeing that new upgrades at the LHCb, Belle and Atlas experiments could mean that despite the flavour sector being constraint the lightest possible Higgs might soon be in the TeV range. 

We recognize these conclusions might not be sufficient to fully examine the model. There might be room for more conclusions if exact alignment is not performed and room for light mixing between scalars is allowed. 
%
%Especially if this mixing comes from light scalars. 
%
%Note that we found that there are exotic scalars with very light masses that sucessful pass all constraints. 
% 
%This study might pave the way for continued work in direct scalar searches. 

%T some of the allowed exotic scalars are found to be light indicating that the model succeeds in confronting flavour data even in the presence of scalars as lightas 300 GeV. Hence, the studied 3HDM opens the door to direct searches for non-SM scalarsat future runs of the LHC.

%As for the 3HDM presented in Chapter\,\ref{ch:3HDM}. 
%
%Our numerical analysis show that a 3HDM Model when stabilized trough a flavour symmetry that is softly broken can provide a way to suppress the appearance of tree-level FCNCs. 
%
%Although these FCNCs mediated by the scalar fields have NP contributions that can be sizable, but are not when scalar scattering is considered. 
%
%We also discuss that there are stronger constraints on the model coming from it's enlarged scalar sector.
%

%\section{ Future Work}

