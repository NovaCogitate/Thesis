%\chapter{Introduction}
%\label{ch:Intro}

\newpage

\chapter{Introduction}
%\section{Introduction}

%\section{The Sturdy SM with some holes}

Modern study of particle physics must be taught trough the Standard Model (SM) of particle physics. 
%
The SM has thus far the best descriptor for the experimentally observed spectra of particles and their interactions at all current probable scales. 
%
And In 2012 a resonance was discovered in the LHC that seems to confirm the existence of it's last predicted particle, the Higgs boson, finally completing the Model and proving the existence of the Higgs mechanism \cite{Aad_2012,chatrchyan2012observation,
collaborations2015combined,collaborations2016measurements}. 

The development of the SM was a arduous task, it led scientists successfully  combine three of the four fundamental forces of nature in a very well motivated framework, making it one of the most monumental achievements in physics.
%
However, despite it's successes the SM still lacks a strong theoretical explanation for several experimental observations. They have become more numerous by the decade, and to provide a "short" overview of some of them. 

We firstly have the fact the SM can not account for one of the most important cosmological discoveries of the century, the existence of dark matter. This is a fundamental flaw since the SM lacks a possible dark matter candidate, or dark particle. 
%
Secondly, the SM lacks any justification for the existence of baryon asymmetry in the universe, i.e. why is the universe primarily made of matter rather than anti-matter. 
%
Although note that the Electroweak baryogenesis (EWBG) remains a theoretically possible scenario for explaining the cosmic baryon asymmetry, a scenario viable in the SM framework.
%
Thirdly, the SM suffers from peculiar oddities in the fermion sector in the form of unjustified mass and mixing hierarchies. This is usually refereed to as the \textit{flavour problem} and is considered a sizeable drawback of the SM. 
%
As a example, we observe the top quark to be five order of magnitudes heavier the up quark, and eleven orders of magnitude than the observed neutrino masses. These high differences are thought to be too large to be natural, so a physical property that would justify such gap is a desired property of most Beyond the Standard Model (BSM) frameworks. 
%
Fourth, note neutrino masses are not included in the SM. Although there are precise oscillation measurements that measure masses in the eV range with precise mixing in between 3 different generations of neutrinos. 
%
There are still many other subtitle flaws, like the lack of a strong phase transition, etc. 

These are just some of the typical justifications given to explore possible BSM scenarios. The holy grail of which would be a model that include all these problems in a properly motivated framework that addresses these and many more cosmological, gravitational, and phenomenological problems.  
% 
For now such a model remains out of reach, so the narrowing down of theories through phenomenological studies is a very worthwhile endeavour. We try to present one of these studies in this work. % to the steady advancement of a more complete theory. 
%
Paradoxically fortunately, as of late these studies have become progressively harder to perform given that the available space for new physics gets reduced by each successful particle experiment. 
%
Chief among these experiments is the Large Hadron Collider (LHC), whose large amount of collect data over past years is setting more and more stringent bounds on viable parameter spaces of popular BSM scenarios. 
%
And as available space for new physics decreases it becomes more challenging to reveal remaining space without falling within the possibility of fine tuning our model.  

%{\color{blue} How to properly explain what fine tunning is? Should I?}

Note, that the SM has shown increasingly consistence with most constraints that were initial believed to be a possible gateway to new physics (NP) or that would diverge from it's predictions. Thus, the search continues for hints at possible directions to complete the SM. % One of these is brought to use trough flavour physics, as we'll soon examine further bellow. 

Conventionally, phenomenological simulations of BSM searches in these multi-dimensional parameter spaces have been made in large computer-clusters with use of several weeks of computational time trough simple Monte-Carlo methods. 
%
Although this is the basis of the work presented here a effort was made to incorporate new machine learning routines trough the initial building of smaller learning sets trough conventional methods. 
%
Unfortunately this wasn't accomplished in this work due to the expectational setbacks. A feature of this year, that affected partially the quality of the work. 

%%%% so far so good

During this work we shall do a small expedition into two possible BSM scenarios.
%
To achieve this, we will start by laying down the fundamental basis for this BSM discussion by presenting a short overview of the SM, then we discuss possible extensions to the SM. Namely, first by presenting the B-L-SM model, a simple unitary extension based on a apparently accidental symmetry of the SM. 
%
And then by moving on to a more complex model with additional Higgs doublets fields as a attempt to present a framework that addresses the \textit{flavour problem}. 
%
We will see how these multiple doublet Models can address problems that simple unitarity extension can't and vice-versa. For example multiple Higgs Doubles can easily offer a explanation for the observed excess of charge parity or $\mathcal{CP}$ violation. But suffer from the possible inclusion of tree-level Flavour Changing Neutral Currents (FCNCs). These FCNCs are undesirable at least in large number given observations, so mechanisms have to be put in place to prevent them, while in the case of the simple unitary extensions such problems do not arise. 

%We will see how these models with more than one Higgs doublet can address yet another, thus far, unmentioned problem in the SM, the observed excess of charge parity or $\mathcal{CP}$ violation.  
%
%While suffering the possible the drawbacks of potentially having large Flavour Changing Neutral Currents (FCNCs). These FCNCs are undesirable at least in large number given observations, although multiple Higgs Doublets could include these diagrams at tree-level, making them very problematic. We present a specific version of a Multiple Higgs Doublet, specifically a 3HDM model with a symmetry mechanism that will suppress these FCNCs. 

I also want to stress that, while the minimally structure of the Higgs sector postulated by the SM is not a immediate contradiction of measurements. It is not manifestly required by the data. And in fact a extended scalar sector is often desired despite the relatively tight bounds on Higgs boson couplings to SM gauge boson and heavy fermions. 
%
These additions are motivated also in part by in the  SM,  the  single Higgs  doublet is  a bit "overstretched".  It  takes  care  simultaneously  of  the masses of the gauge bosons and of the up and down-type fermions and leptons. N-Higgs-doublet models and scalar or complex fields,  relax  this  requirement.   
%
In particular the multiple Higgs doublet models are based "natural",  suggestion,  that the  notion  of  generations  can  be  brought  to  the  Higgs  sector. 

%Both these extensions have the bonus of leading to remarkably rich  phenomenology (for a detailed review, see e.g. Refs. ( \cite{branco1999cp,Branco_2012,Ivanov_2017} ). And in general BSM scenarios offer features  as  several  Higgs  bosons,  charged  and neutral, modification of the SM-like Higgs couplings, FCNC at tree level, additional forms of CP-violation from the scalar sector, and opportunities for cosmology such as scalar DM candidates and modification of the phase transitions in early Universe. {\color{blue} Repeated, Fix later.} Also, many BSM models including supersymmetry (SUSY), gauge unification models, and even string theory constructions naturally lead to several Higgs doublets at the electroweak scale.

%We give a higher repute to the Higgs Sector since fermion masses and mixing patterns relate often to the specific structure of the Higgs sector. Also, the addition of new scalars offer a large playground for collider experimentation and often offer the inclusion of new neutrino physics. 

%In short two particular multi-Higgs models will be presented in this work a phenomenological study of a 3 Higgs Doublet model (3HDM) with softly broken $\U \times \mathrm{Z_2}$ symmetry and a simple Unitarity, $\mathrm{U(1)}$, extension of the SM based on the apparent Baryon minus Lepton symmetry (B-L-SM). We'll investigate what can be learned from these models and what other physical experiments constrict them. 

%The SM extensions featuring non-minimal Higgs sectors with extra Higgs doublets in analogy to fermion generations in the SM provide a fruitful playground for constructing successful BSM scenarios (for a detailed review, see e.g. Refs. \cite{branco1999cp,Branco_2012,Ivanov_2017} ).

%There is also no constraints stemming from the $\rho$ parameter here.  Since all doublets couple to the gauge-bosons in the same way, the W and Z masses are determined by the single value, the sum of real VEVs. Assuming this value is 246 GeV it would retain the condition $\rho = 1 $ at tree-level. 

%Multi-doublet models offer novel opportunities for CP-violation.  Within the SM, it is put by hand coming entirely from the Yukawa matrices which must be complex.  In multi-doublet models, a relative phase between vevs can arise just as a result of the minimization of the potential. Leading to a more natural and spontaneous CP-violation. 

%A real or even complex singlet extension is a also simple pathway to extending the SM. In a generic model with a SM Higgs Doublet the addition of a generic gauge singlet scalar, S, could prove a link between the SM fields and a unknown hidden sector. 
%
%In spite of our ignorance of this hidden sector, we can simply assume a generic renormalizable self-interaction for the scalar S and investigate the joint $(\phi,S)$ potential. This would lead to a generic mixing, $\alpha$ between scalars. 

%In the case the additional scalar field is complex this brings a additional degree of freedom and has the possibility of 3 neutral scalars mixing, depending on the shape of the VEV. Producing  a slightly  richer  collider  phenomenology  and  complicating its analysis.  

%Are heavily constricted from experimental measurements we know that in this framework fermions and gauge bosons should primary couple to $h_\phi$. This type of mixing suppression, ($\alpha < 10^3$), but even so the heavy Higgs can in most cases decay into a pair of light ones if this channel is open kinematically, providing a avenue for detection. 

%On the other hand, one or both new scalars can be symmetry protected against decay, yielding simple models of one or two-component dark matter or models with one DM candidate and a strong electroweak phase transition.

%Before moving on, let us make a remark on the (absence of) CP-violation in the singlet extension of SM. Although the potential contains many complex coefficients, it does not produce CP-violating effects in the scalar sector, see \cite{branco1999cp}.




