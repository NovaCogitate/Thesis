
Any 3HDM can be thought of as part of a larger family of multiple Higgs Doublet Models, or NHDMs, the first iteration of which was the Two Higgs Doublet Model (2HDM) proposed by T.D. Lee \cite{Lee1973}. At the time, Lee was motivated by the search for a spontaneous breaking of the CP symmetry which was included in his model.  However the 2HDM quickly became a very popular model motivated by it's inclusion for dark matter candidates, as well as providing a large particle spectrum, including charged and additional neutral scalars who enable FCNC deviations. In spite of the fact that, now, these tree-level FCNCs are in direct opposition to experimental results, as discussed in section \ref{Chap_1_Sec_3}. Given these limits, we can consult the literature, as in Ref. \cite{Branco:1999fs}, and see that the existence of FCNCs forces the extra scalars in the minimal 2HDM case to have masses above 1 TeV as to suppress FCNCs. These heavy scalars are far from ideal, since there is no indication such heavy scalars exist nor do they provide us with interesting physics. 

Several mechanisms have been proposed to deal with suppression of these tree-level FCNCs as to allow for richer physics and lighter scalars. First, in Ref. \cite{ferreira2019strong}, it is proposed a framework in which we have the balancing of CP-odd and CP-even contribution to FCNCs, however, this requires some fine-tuning, making it very unappealing based on a ``naturalist" argument. Another possibility is to assume alignment between different Yukawa matrices such that no FCNCs are present, see \cite{Jung_2011}. Finally we could also use the approach presented in the BGL version of the 2HDM \cite{Branco_1996}, here the authors impose a flavor-violating symmetry naturally keeping the FCNCs under control trough the CKM matrix. The phenomenology of the model has been studied quite thoroughly in previous works, see Ref. \cite{Botella_2016}, and it remains a possible scenario for BSM physics.

As the name indicates, NHDMs, are types of models where, in parallel with the standard SM Higgs doublet some additional replicas of that same  (or slightly modified) doublet are introduced. In the NHDM these form a sort of family in the scalar sector in analogy to the fermion sector. Being that the 3HDM is the most similar. This idea is far from original and was first discussed by Weinberg in Ref. \cite{Weinberg1976}.

Phenomenologically speaking these 2HDMs and 3HDMs models are quite different however, and we can argue that there are several advantages to the 3HDM. Such as, unlike in the 2HDM case, the 3HDM can provide more than one stable source of spontaneous CP symmetry breaking \cite{Branco_2012}. Furthermore, charge breaking minima were found to be stable while at the same time coexisting with charge-preserving ones (for more information see, \cite{Barroso_2006}). Note also, for the 2HDM a full list of all possible incorporation of symmetries consistent with $\mathrm{SU(2)}\times\mathrm{U(1)}$ has been achieved \cite{Ivanov_2008}, while for the 3HDM no work has thus far been completed, see, \cite{Ivanov_2015}. 

In this work we explore a 3HDM with a global flavor symmetry to replicate the BGL treatment. And in fact similar studies have been done in 3HDMs that include different flavor symmetries, such as Ref.\,\cite{Camargo_Molina_2018}. Given this background it is no surprise that we will focus particularly in the measuring Quark Flavor Violation (QFV) observables. In our analysis these will consist of meson decays or oscillations that include the changing of quark flavor within the meson. These QFV observables will be compared to measured amplitudes in collider experiments as to offer a novel exclusion to our model. Note, the additional flavor symmetry constrains the terms that can appear in the Yukawa sector of the Lagrangian resulting in very specific structures (or textures) of the Yukawa couplings. These shapes generate a suppression mechanism trough the CKM matrix off-diagonal terms restricting the values of FCNCs in our model.%

