

% 
% The T parameter measures isospin violation, since it is sensitive to the difference between the loop corrections to the Z boson vacuum polarization function and the W boson vacuum polarization function. An example of isospin violation is the large mass splitting between the top quark and the bottom quark, which are isospin partners to each other and in the limit of isospin symmetry would have equal mass. 

% The S and T parameters are both affected by varying the mass of the Higgs boson (recall that the zero point of S and T is defined relative to a reference value of the Standard Model Higgs mass). Before the Higgs-like boson was discovered at the LHC, experiments at the CERN LEP collider set a lower bound of 114 GeV on its mass. If we assume that the Standard Model is correct, a best fit value of the Higgs mass could be extracted from the S, T fit. The best fit was near the LEP lower bound, and the 95% confidence level upper bound was around 200 GeV.[1] Thus the measured mass of 125-126 GeV fits comfortably in this prediction, suggesting the Standard Model may be a good description up to energies past the TeV ( = 1,000 GeV) scale.

Thesis script. 

Capa: 

Hello my Name is faggot mc Faggson. I am here to present my thesis work. I performed a phenomenological scan through a set of scripts I developed on 2 beyond the standard model models under the supervision of Antonio Morais. 

General Structure:

The general structure of this presentation will be as follows: first, we’ll look over some basic concepts of the Standard Model, which we will call the SM for shortness, this is done so we can touch on flavour physics. Secondly we move on to the B-L-SM model. A simple unitary extension of the SM. And we finalize with a more complex model, the 3HDM. 

Standard Model:
So beginning with the Standard Model. 

Modern Physics: 
We know the SM is a very successful theory in particle physics that somewhat recently had its last particle discovered. However we should highlight that evidence is mounting against it. To touch on some of its flaws the SM, cannot for example include dark matter or neutrino masses on it’s structure. These more conceptual flaws create the need for new physics, physics which particle physicists are searching for in large experiments like the LHC. 

Slide 1: 
The SM is a standard Quantum field theory, this means that all it’s dynamics are the result of quantum fields, and that its most fundamental object is the Lagrangian. This should be common knowledge so we are going to speed along some definitions in this section and show immediately the resulting Lagrangian.



B-L-SM:
Ok so now we move on to the B-L extension of the SM. 

Slide 1:
So what is the B-L-SM? The B-L-SM is essentially a SM extended with a new unitary group based on an accidental B-L symmetry.  Thanks to this new symmetry we can address both cosmological, grand unification and phenomenological problems. Meaning this is a simple model with deep implications. 

Slide 2: 
The key differences from the SM are, the new Z’ and heavy scalar and the presence of 3 right handed neutrinos, these allow for dark matter through the new particles or 
